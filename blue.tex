\documentclass[11pt]{book}
\usepackage[usenames,dvipsnames]{color}
\usepackage{array}           % For double-column exercises.
\usepackage{longtable}       % For double-column exercises.
\usepackage[normalem]{ulem}  % For strikethrough.
\usepackage{paralist}
\usepackage{enumitem}        % To "resume" enumerated lists.
\usepackage{epsfig}
\usepackage{afterpage}
\usepackage{multicol}
\usepackage{fancybox}
\usepackage{makeidx}
\usepackage{textcomp}
\usepackage{fancyvrb}
\usepackage{footnote}
\usepackage{makecell}
\usepackage{amsmath,amsthm,amsfonts,amssymb,latexsym}
\usepackage{MnSymbol}
\usepackage{wasysym}
\usepackage[width=.9\textwidth,singlelinecheck=true,skip=1pt,font=small,labelfont=bf]{caption}
\newcommand{\freakingtilde}{\raisebox{0.5ex}{\texttildelow}}
\makesavenoteenv{tabular}  % to allow footnotes in tables
%\captionsetup{width=.9\textwidth}

%\usepackage{enumerate}

\makeindex

\newenvironment{custommargins}[2]%
  {\addtolength{\leftskip}{#1}\addtolength{\rightskip}{#2}}{\par}

% Set left margin - The default is 1 inch, so the following
% command sets a 2-inch left margin.
\setlength{\oddsidemargin}{1in}
\setlength{\evensidemargin}{0in}
% Set width of the text - What is left will be the right margin.
% In this case, right margin is 8.5in - 1.25in - 6in = 1.25in.
\setlength{\textwidth}{5.5in}

% Set top margin - The default is 1 inch, so the following
% command sets a 0.75-inch top margin.
\setlength{\topmargin}{.75in}

% Set height of the text - What is left will be the bottom margin.
% In this case, bottom margin is 11in - 0.75in - 9.5in = 0.75in
\setlength{\textheight}{7.25in}

\setlength{\parindent}{0pt}
\setlength{\baselineskip}{1.5pt}
\setlength{\parskip}{6pt}

\begin{document}

\title{Blueprints: Creating, Describing, and Implementing\\Designs for Larger-scale Software
Projects\\{\small version 1.0.0}}
\author{Stephen Davies, Ph.D.\\Computer Science Department\\University of Mary Washington}
\date{}
\maketitle

\frontmatter

\renewcommand{\contentsname}{Contents at a glance}

\setcounter{tocdepth}{0}
\tableofcontents

%\include{preface}

\setcounter{chapter}{0}

\mainmatter

\chapter{Getting off the ground}
\label{ch:gettingOff}

\index{command line}
\index{Linux}
\index{Unix}
Before we begin our study of object-oriented systems proper, we'll introduce
the command-line toolset we'll be using to construct our programs. We'll take
each of the most important tools out of our toolbox, lay them out before us on
a little mat, and learn what they're for.

\section{Why the command line?}

\index{CLI (command-line interface)}
\index{GUI (graphical user interface)}
Developing software in a command-line environment (sometimes abbreviated
``CLI'' for \textbf{command-line interface}, as opposed to a
``GUI\footnote{Commonly pronounced ``gooey.''}'' or \textbf{graphical user
interface}) involves typing white text in little black boxes. It requires
memorizing and regurgitating a variety of obscure commands. It demands exact
adherence to an inconsistent syntax, and exacts heavy penalties for mistakes,
all while providing only a very crude and clunky-looking interface.

It's natural to wonder why we would want to do this. After all, aren't
computer systems immeasurably more sophisticated now? If even end users run
fancy, graphical, forgiving apps, shouldn't computer scientists expect even
easier-to-use and sexier-looking stuff?

It may seem so, and in terms of the \textit{power} the tools provide, we'll
discover that indeed software developers are aptly equipped. But in some ways
it's a false expectation to assume that our toolset would be as \textit{easy}
to operate as that of an everyday user. After all, which is easier: to drive a
car, or to be a mechanic? Even though I enjoy cruise control and
auto-adjusting seats, I don't find it strange at all to learn that mechanics
still use socket wrenches to adjust piston assemblies.

Much of what makes a CLI so powerful is its expressiveness. A driver can press
any of the three or four cruise control functions the manufacturer provided.
But a mechanic can take any of hundreds of tools, tweak dozens of different
parts, and combine these adjustments in uncountable ways. That's the kind of 
flexibility the command line provides.

The difference between a CLI and a GUI is that with the latter, the user can
essentially do \textit{only what the tool designer anticipated she would want
to do.} There's no way she can express something that isn't one of the
tailor-made menu options.

When you use the command line, think of it as composing sentences, word by
word. A GUI comes with a repertoire of standard sentences you can choose from.
That makes it easy to do standard things, and hard to make silly mistakes. But
a CLI, being inherently language-based, is immeasurably more flexible. You can
write any (legal, grammatically correct) sentence you choose, even one the
designers of the CLI never thought of, and even one that you didn't know you'd
want to type until a moment ago. The bits and pieces can be combined in a
myriad of ways, just as nouns and verbs can.

\index{Finlayson, Ian}
There are other reasons as well that many developers live on the command line.
Among them are\footnote{Thanks to Ian Finlayson for capturing much of this
list.}:

\begin{itemize}
\itemsep.1em

\item \textbf{Speed.} It turns out to be way, way faster to type commands --
in combination with the various shortcuts and recall/edit operations -- than
it is to sift through menu options and such with a mouse. Trust me.

\index{remote access}
\index{shell}
\item \textbf{Remote access.} When you're running programs on your own device,
it's possible to do it with a GUI. But computer scientists very often have to
connect over the network to distant machines in order to tell them what to do.
Every time you need to configure a web server, for instance, or update a
publicly-accessible database, or run a time-consuming job on a parallel
cluster, or correct the data on your mobile device, you need a way to issue
commands to another machine through a very low-bandwidth channel. Opening a
command line ``shell'' to that remote device is by far the most common and
effective way to do this.

\index{scriptability}
\item \textbf{Scriptability.} There's just no good way to automate a sequence
of GUI operations. To explain to someone else how to accomplish something, you
have to painfully walk them through each operation (``go to the Start menu and
find Accessories, then in the Math menu choose Calculator...when it comes up,
right-click in the background and enable Advanced Options...'') which is
tedious and error-prone. It'd be nice if you could just send them a custom
command which would do all that. As a matter of fact, it would be nice if
\textit{you} could make a custom command which would do all that, so that you
could execute it many times without rehashing the same rigmarole. You'll find
that CLIs are eminently automatable in this way. You can create custom
commands called ``scripts'' that are combinations of other interacting
commands, and in this way you become master of your whole world.

\index{Cygwin}
\index{Mac OS X}
\index{Microsoft!Windows}
\index{Raspberry Pi}
\index{Kindle}
\index{Terminal application}
\item \textbf{Consistency.} Graphical user interfaces are more different from
each other than CLIs are. Partly this is because nearly any CLI you're likely
to use is Unix/Linux-based\footnote{For our purposes, you can consider the
terms ``Unix'' and ``Linux'' exact synonyms. The Mac OS X command line
(available through the ``Terminal'' app) is Unix-based, too. Windows machines
aren't, but programs like ``Cygwin'' can be downloaded for free and provide a
Linux-like command-line veneer over the operating system.}, and hence they all
``speak the same language.'' It's great to be able to log on to different
laptops, web servers, your phone, your Kindle, or a Raspberry Pi and get the
same prompt that understands the same stuff.

\index{Linux}
\index{Unix}
\index{CLI (command-line interface)}
\item \textbf{Stability.} CLIs rarely change. When they do, it's very very
rarely in a non-backwards-compatible way. By contrast, every time a new
graphical user interface is released, you have to go through a period of
hunting around and finding out where everything is. With Unix/Linux, you can
literally run commands that were written last century and they will likely
still work as is.

\end{itemize}

There's always a few students who, despite the above benefits, resist learning
this material at first. I get it. It's like learning a new language, and the
immense effort to understand an alien world sure doesn't feel like it's going
to pay off any time soon. All I can say is that if you're not convinced it's
worth it, for now just think of it as something you have to master ``just
because your professor and the industry says so.'' My hope is that by the end
of this course, you're pleasantly surprised by seeing some payoff for your
hard work.


\section{The filesystem}
\index{filesystem}
\index{file}
\index{tree}
\index{directory}
\index{folder}

Okay. The backdrop for all our use of the Linux command-line interface is the
\textbf{filesystem}.\footnote{Often, not always, written as a single word as I
have it here.} Any general-purpose computer, no matter its architecture or OS,
has an area of permanent storage for user data. Interestingly, and
conveniently, all computers organize their filesystems in pretty much the same
way: as a \textbf{tree} of \textbf{files} and \textbf{directories}.
(Windows/Mac users will be familiar with the term ``\textbf{folder},'' which
\textit{means exactly the same thing as ``directory.''}) In what follows, we'll
be using a different syntax (text instead of visual icons) to work with what is
conceptually the same organizational structure you're used to on your own
computer.

\subsection{Files and directories}

\index{filesystem extension}
A file is simply any named chunk of stuff on your disk. Images, .mp3 tunes,
Word docs, and (importantly) plain text files are all in this category. On
Windows, you're used to each of these files having a filesystem ``extension''
designating its type: ``\texttt{.docx}'' means a Word doc, and
``\texttt{.jpg}'' means an image file, for example. This is sort of true with
Linux, although the rules are a bit looser. Not all files have extensions at
all, and when they do, it's more a signal that they're intended to be treated a
certain way than it is a hard-and-fast requirement.

\index{java file@\texttt{.java} file}
\index{source file}
\index{Microsoft}

The most important files you'll work with in this class will have a
\texttt{.java} extension. These are your Java \textbf{source files}. You'll
also work with other various supporting files to make all the tools work
correctly. It's important to realize that \textit{a file is fundamentally just
some data, which can theoretically be opened and dealt with by any program}.
When we say that \texttt{HamletPaper.docx} ``\textit{is}'' a Word doc, what we
really mean is that its data is formatted in a certain way that the Microsoft
Word application expects to see, so it can render it on the screen for editing.
But it is possible to open that same \texttt{HamletPaper.docx} file with other
programs and manipulate its contents. This may seem sketchy, but it is actually
a force for good.

\index{text file}
In particular, you'll be tempted this semester to think of a \texttt{.java}
file as ``a \texttt{vim} file,'' in the same way that you may think of an
\texttt{.xls} file as ``an Excel file.'' I hope to break you of this habit, as
you learn to see a file as text or data that is actually independent of what
kind of program might be used to open and manipulate it.

\index{directory}
A directory is a container for files \textit{and also other directories}. That
last italicized phrase is what gives rise to the overall tree structure of the
filesystem, as discussed in the following section.

%\index{Foreman, George}
By the way, every file and directory \textit{in a particular directory} must
have a unique name. You can't have a file called ``\texttt{DireStraits.mp3}''
and another one also called ``\texttt{DireStraits.mp3}'' sitting there in the
same folder: it's a name collision. However, it's perfectly permissible to have
two files with the same name in different directories. This is kind of like how
there isn't more than one ``Stephen'' in my immediate family (that would be
confusing\footnote{With apologies to boxing legend George Foreman, who named
all four of his children ``George.'' That practice is not
filesystem-compatible.}), but there are of course many ``Stephens'' in the
world.

\subsection{The filesystem tree}

\index{tree}
\index{filesystem}
\index{directory!parent}
\index{parent directory}
The files and directories in a filesystem form a nested, hierarchical
structure called a \textbf{tree} (see Fig.~\ref{fig:tree}). I have drawn two
kinds of nodes in this tree: directories (yellow ovals) and files (blue
boxes). As expected, some of the directories have arrows coming out of them,
but none of the files do. The elements that a directory is pointing to are the
contents it contains: \textit{e.g.}, the left-most ``\texttt{america}''
directory contains another directory (``\texttt{nation}'') and also the file
\texttt{A.txt}. We use the term \textbf{parent directory} to mean the
directory immediately above an entry in the filesystem; the left-most
\texttt{A.txt} file's parent directory is the \texttt{america} directory we
just spoke of.

\begin{figure}[ht]
\centering
\includegraphics[width=0.9\textwidth]{tree.png}  %650x300
\caption{The Linux filesystem, in pictorial form.}
\label{fig:tree}
\end{figure}

In order to keep you on your toes, I've given several entries in this example
filesystem the same name: in addition to a couple different \texttt{america}s,
we've got several \texttt{states}, multiple different \texttt{A.txt}s, etc. In
no case, however, are the duplicately-named entries in the same directory.
(Convince yourself of that fact.)

\subsubsection{Only one surprise}

\index{Microsoft!Windows}
So far this is pretty easy. And it won't get much harder. But here's the one
thing you have to get used to: with a CLI, \textit{we won't ever actually see
that filesystem picture visually.} It's there, but we don't explicitly view it
in graphical form. Instead, there will be a textual way of referring to every
file and directory. It's straightforward, but can be a bit of a shock to those
coming from point-and-click systems like Windows.

\subsection{The ``current'' (or ``working'') directory}

\index{directory!current}
\index{directory!working}
\index{current directory}
\index{working directory}
One vital concept to grasp is that every time we issue a command or run a
program in Linux, we are doing so \textit{within the context of a particular
directory}. Conceptually, we think of being ``in'' a certain directory at any
point in time. We call this directory ``the \textbf{current directory}'' or
``the \textbf{working directory}'', and we'll learn commands to find out what
it is and to change it to something else.

Which one we're ``in'' has a crucial impact on what happens when we execute a
command. For instance, if our current directory is the far-left
\texttt{america} directory, and we issue a command that does something to
``\texttt{A.txt}'', it would act on the left-most \texttt{A.txt} file, since
it's the one within the current directory. But if our current directory were
\texttt{unitedstates}, ``\texttt{A.txt}'' would instead mean the far-right blue
node.

I've found that failure to understand the ``current directory'' concept is one
of the most common trouble spots for beginning Linux programmers.

\subsubsection{The root directory}

\index{directory!root}
\index{root directory}
\index{tree}

Okay, back to the filesystem as a whole. At the top of the tree is the
\textbf{root directory}, which has no parents. (This is often disorienting to
non-computer-scientists, since in the real world you may have noticed that
trees actually grow \textit{up}, not down. But in computer science, we always
draw trees growing down from the root.)

\index{backslash}
\index{forward slash}
\index{slash}
\index{Microsoft!Windows}
The root directory is the anchor point of the entire filesystem: it ultimately
contains everything under it. It also has a very strange name: ``\texttt{/}'',
pronounced ``slash.'' (This is a ``forward slash,'' by the way, to the left of
your right-most Shift key, not a ``backslash.'' Oddly, most Windows systems
use a backslash ``\texttt{\textbackslash}'' for this instead.) Stay awake,
because this ``\texttt{/}'' character will shortly mean something very
different as well.

\subsubsection{Paths}

\index{path}
\index{directory}
\index{file}
It should be apparent to you that as a consequence of this nested tree
structure, you can ``reach'' every element from the root directory by
traversing from arrow to arrow. Furthermore, you can do so in only one way.
For instance, the \texttt{B.java} file can be reached from the root by going
from ``\texttt{/}'' to \texttt{unitedstates} to \texttt{B.java}. And that's the
\textit{only} way to get there. You can reach \texttt{loveLetter.txt} by going
from ``\texttt{/}'' through \texttt{united}, \texttt{states}, and
\texttt{america}, in that order. This is true for every file and directory.

What this means is that every entry has a unique \textbf{path}, and we can
express it in text as well as in a diagram. Take the \texttt{B.java} file for
example. Its path is:

\quad\quad \texttt{/unitedstates/B.java}

\index{slash}
Look very carefully at that string as we dissect it. The most important thing
to grasp is that the two slash (``\texttt{/}'') characters \textit{each mean
something different}. The first one means ``the root directory, which is called
slash.'' But the second one is merely a separator, delimiting the
\texttt{unitedstates} from the \texttt{B.java}. So this path means
``start at the \textit{root} directory, go down to its \texttt{unitedstates}
entry (which itself is a directory), and there you have the \texttt{B.java}
file.''

Similarly, the path to \texttt{loveLetter.txt} is:

\quad\quad \texttt{/united/states/america/loveLetter.txt}

\index{slash}
(Note that the slash between \texttt{united} and \texttt{states} makes all the
difference in the world: if it weren't there, we'd be starting our descent
through the right-most \texttt{unitedstates} directory as before.)

\index{path!absolute}
\index{absolute path}
\index{planet}
\index{path!relative}
\index{relative path}
These paths are called \textbf{absolute paths} because they \textit{start with
a slash}. This means that they give the complete, start-from-the-top position
of a particular file or directory. It's kind of like referring to a building
by its complete address, including city, state, zip code, country, and planet.
Often we want a short-hand way of referring to an entry without specifying its
entire absolute path. To do so, we use a \textbf{relative path}.

\index{directory!current}
\index{current directory}
A relative path is relative \textit{to the current directory}. And it does
\textit{\textbf{not}} begin with a slash. Instead, it gives directory names,
separated by slashes, indicating where to start descending from the
\textit{current} directory.

For example, let's say the current directory was ``\texttt{/states}''. And
suppose I used this relative path:

\quad\quad \texttt{united/states/mysteryNovel.txt}

(Note carefully that it has \textit{no} initial slash!) This relative path
would start at the \textit{current} directory (\texttt{/states}) and from there
traverse down to \texttt{united}, \texttt{states}, and then finally
\texttt{mysteryNovel.txt}. Obviously, where you end up is critically dependent
on where you start -- on what the current directory is.

\index{directory!current}
\index{current directory}
To test your understanding, realize that in this case where the current
directory is \texttt{/states}, there is no such file called
\path{united/states/america/loveLetter.txt}. In fact, even 
\path{united/states/america} doesn't exist. However, if we changed the
current directory to be the root (``\texttt{/}''), suddenly the relative path
\path{united/states/ america/loveLetter.txt} would be legit.

\section{Linux A-B-C's}

\index{Unix}
\index{Linux}
\index{filesystem}
\index{file}
\index{directory}
With the filesystem always hovering in the background, let's introduce the
first basic Linux commands to work with the files and directories. These
commands are so basic that they're like the alphabet of speaking any Linux
sentence. Using them should eventually be as familiar and effortless to you as
clicking the mouse.

\index{prompt}
\index{dollar sign (\texttt{\$})}
In all that follows, I will precede anything you are to type on the Linux
command line with a dollar sign \textbf{prompt}:

\begin{Verbatim}[fontsize=\small]
$
\end{Verbatim}

To execute a command, you do \textit{not} type the prompt itself: it's just
there to indicate ``now is an appropriate time and place to enter a Linux
command.'' Just type the stuff after it.

\index{directory!current}
\index{current directory}
Also, depending on your system and configuration, your prompt may look
different or have other information in it. One common setting, for instance,
is for the current directory to always appear as the prompt. (I personally
hate that, since it makes different commands start at different horizontal
locations as I work, plus it consumes a lot of space.) No matter what, though,
just mentally substitute ``the dollar sign'' for ``whatever your Linux prompt
is.''

\newcommand{\bigline}{\vspace{-.3in} \begin{center} \line(1,0){300} \end{center}}


\begin{enumerate}
\itemsep.1em
\item \textbf{\texttt{pwd}}

\index{pwd@\texttt{pwd}}
Your first command stands for ``\textbf{p}rint \textbf{w}orking
\textbf{d}irectory,'' and simply tells you what the current directory is at any
point in time. For example:

\begin{Verbatim}[fontsize=\small]
$ pwd
/united/states
\end{Verbatim}

tells you that you're currently ``in'' the \texttt{states} directory, which is
contained within the \texttt{united} directory, which is contained within the
root directory.

\textbf{Tip:} get in the habit of typing \texttt{pwd} a \textit{lot},
especially at first. Get ingrained in your brain the question ``where am I in
the filesystem right now?'' because it matters, yet is not in your face except
when you type this command.

\bigline
\item \textbf{\texttt{cd}}

\index{cd@\texttt{cd}}
\texttt{cd} stands for ``\textbf{c}hange \textbf{d}irectory'' and is how you
move to another place. You give it an \textbf{argument} (kind of like passing
a parameter to a function call in a programming language, although we don't
use parentheses or commas here) which is where you want to go:

\begin{Verbatim}[fontsize=\small]
$ cd  /america/nation
\end{Verbatim}

Here, I've specified an absolute path. If I now execute \texttt{pwd}, I see
that it worked:
\begin{Verbatim}[fontsize=\small]
$ pwd
/america/nation
\end{Verbatim}

\index{relative path}
\index{path!relative}
\index{absolute path}
\index{path!absolute}
More common is to specify a relative path. If we first go back to our original
location:

\begin{Verbatim}[fontsize=\small]
$ cd  /united/states
$ pwd
/united/states
\end{Verbatim}

we can then say ``go \textit{from here} into the \texttt{america} directory'':

\begin{Verbatim}[fontsize=\small]
$ cd  america
$ pwd
/united/states/america
\end{Verbatim}

\index{cd@\texttt{cd}}
\index{pwd@\texttt{pwd}}
I can't overestimate how important it is to notice that in the previous
\texttt{cd} command, I did \textit{not} include a slash before
\texttt{america}. If I had, it would have been an absolute path, and I would
have gone to a completely different part of the filesystem:

\begin{Verbatim}[fontsize=\small]
$ cd  /america
$ pwd
/america
\end{Verbatim}

\subsubsection{``Special'' directory shortcuts}

\index{directory!shortcuts}
\index{special directory shortcuts}
This is a good time to mention that when you are specifying paths, there are
three very common shortcuts that you'll want to know about.


\begin{itemize}
\itemsep1em

\index{directory!current}
\index{current directory}
\item The current directory: \texttt{\textbf{.}}

A plain-ol' dot (period) is used to mean ``the current directory.'' There's no
obvious uses for this yet, but believe me, it comes up \textit{all} the time,
so just memorize it. A useless example for now:

\begin{Verbatim}[fontsize=\small]
$ pwd
/united/states
$ cd  ./america
$ pwd
/united/states/america
\end{Verbatim}

So ``\texttt{./america}'' is another way of saying ``\texttt{america}''. (Told
you this example was useless.)

\item The parent directory: \texttt{\textbf{..}}

\index{directory!parent}
\index{parent directory}
More immediately useful is the double-dot, which means ``the parent of the
current directory.'' If we're currently in \path{/united/states} and want to go
to \texttt{/united}, one way to do it is:

\begin{Verbatim}[fontsize=\small]
$ cd  ..
$ pwd
/united
\end{Verbatim}

We can also join this with additional relative path stuff to move around the
hierarchy in various ways:

\begin{Verbatim}[fontsize=\small]
$ pwd
/states/united
$ cd  ../usa
$ pwd
/states/usa
\end{Verbatim}

\index{directory!sibling}
\index{sibling directory}
\index{directory!child}
\index{child directory}
Here we went to a ``sibling'' directory by ``going up one, and then down to a
different child.''

\item The home directory: \texttt{\textbf{\freakingtilde}}

\index{directory!home}
\index{home directory}
A shortcut for ``the home directory'' (which means ``the current directory when
you first log in'') is a tilde. It's commonly used in conjunction with other
relative path stuff, like the last double-dot example, above.

Your home directory will probably be something like \texttt{/home/joeschmo}
(which you can verify by just typing \texttt{pwd} when you first log in).
Suppose it is. Then, you can use the tilde:

\begin{Verbatim}[fontsize=\small]
$ pwd
/somewhere/else/in/the/filesystem
$ cd  ~/shortStories/scifi
$ pwd
/home/joeschmo/shortStories/scifi
\end{Verbatim}

to go to any of your subdirectories.

\end{itemize}

\bigline
\item \textbf{\texttt{ls}}

\index{ls@\texttt{ls}}
While \texttt{pwd} tells you what the current directory is, the \texttt{ls}
command (which sort of stands for ``\textbf{l}i\textbf{s}t'') gives you its
contents. If I type it while in the \texttt{/america} directory, for instance,
it tells me:

\begin{Verbatim}[fontsize=\small]
$ ls
nation   A.txt
\end{Verbatim}

These are the two entries from Figure~\ref{fig:tree}.

\index{ls@\texttt{ls}}
A few gotchas to be aware of. First, there's no way from that listing to
tell that \texttt{nation} is a directory whereas \texttt{A.txt} is a file. If
you want to see that, you need to add the ``\texttt{-l}'' option (a minus sign
followed by the lower-case letter ``ell''):

\begin{Verbatim}[fontsize=\footnotesize]
$ ls  -l
-rw-r--r-- 1 kyloren   sithlords   17 Sep  5 16:21 A.txt
drwxr-xr-x 2 kyloren   sithlords 4096 Sep  5 16:21 nation
\end{Verbatim}

Lots of clutter here. The key points:

\begin{compactitem}[-]
\item The far-left character on each line is either a ``\texttt{-}'' or a
``\texttt{d}'', indicating file or directory.
\index{file!owner}
\item Files in Linux have ``owners,'' meaning specific users who created them
and have permissions to manage them. Both of these entries are evidently owned
by user \texttt{kyloren}.
\item The \texttt{17} and \texttt{4096} are file sizes (in bytes).
\item You can see the date and time each entry was last modified.
\end{compactitem}

\index{long file listing}
\index{option}
The ``\texttt{-l}'' stands for ``\textbf{l}ong file listing.'' Most Linux
commands have a bevy of different options you can specify when you execute
them, most often beginning with a minus sign.

Another important one for the \texttt{ls} command is ``\texttt{-a}'' which
stands for ``\textbf{a}ll files, please.'' If that sounds like a strange
option, that's because it is. It turns out that \texttt{ls} by default doesn't
show you all the files; in particular, \textit{it omits those whose names start
with a dot (.).} Why? There are reasons. The only time this will be relevant to
you soon is if you want to work with your \texttt{.bashrc} file.\footnote{If,
in your \textit{home} directory, you create a file with \texttt{vim} called
literally \texttt{.bashrc} (pronounced ``dot-bash-arr-see'') then whatever
Linux commands it contains will be executed automatically every time you log
in. Once you get proficient with Linux, it's very handy to put shortcuts,
aliases, and various preferences in it.} You'd have to type ``\texttt{ls -a}''
in your home directory to actually see it in the listing.


\bigline
\index{cd@\texttt{cd}}
\index{pwd@\texttt{pwd}}
\index{ls@\texttt{ls}}
\vspace{-.1in}
\begin{tabular}{m{.5in}m{3in}}
{\Huge \leftthumbsup} & The above three commands -- \texttt{pwd}, \texttt{cd},
and \texttt{ls} -- go together like Luke, Han, and Leia. Get in the habit of using them literally every minute you're working on the Linux command line.
\end{tabular}
\vspace{.1in}
\bigline

\item \textbf{\texttt{mkdir}}

\index{mkdir@\texttt{mkdir}}
To create a directory in the first place, use the \texttt{mkdir} command and
give it the name:

\begin{Verbatim}[fontsize=\small]
$ mkdir  evilplans
\end{Verbatim}

This new \texttt{evilplans} directory will be created inside the current
directory.

Note carefully that \textit{making a directory does not automatically put you
in it!} Lots of beginners mistakenly think this will happen, but you can see
that it does not:

\begin{Verbatim}[fontsize=\small]
$ pwd
/home/kyloren
$ mkdir  evilplans
$ pwd
/home/kyloren
\end{Verbatim}

\index{cd@\texttt{cd}}
You have to \texttt{cd} as a separate step if you want to now be \textit{in}
\texttt{evilplans}:

\begin{Verbatim}[fontsize=\small]
$ cd  evilplans
$ pwd
/home/kyloren/evilplans
\end{Verbatim}

\index{directory!parent}
A useful option to \texttt{mkdir} is the ``\texttt{-p}'' option which means
``make all \textbf{p}arent directories as necessary.'' This lets us create a
deeply-nested structure all in one fell swoop:

\begin{Verbatim}[fontsize=\small]
$ mkdir  -p  find/luke/skywalker/now
$ cd  find/luke/skywalker/now
$ pwd
/home/kyloren/find/luke/skywalker/now
\end{Verbatim}

\bigline

\item \textbf{\texttt{cp}}
\index{cp@\texttt{cp}}
\index{file!copying}

To make a copy of a file, use \texttt{cp} and give it \textit{two} arguments,
a source and a destination. If I type:

\begin{Verbatim}[fontsize=\small]
$ cp  A.txt  Q.txt
\end{Verbatim}

I will now have two exact copies of the file which can be independently
modified:

\begin{Verbatim}[fontsize=\small]
$ ls
nation   A.txt   Q.txt
\end{Verbatim}

I can also use this to make a (same-named) copy of a file to a different
location, by providing a directory as the second argument:

\begin{Verbatim}[fontsize=\small]
$ cp  A.txt  /states/usa
$ cd  /states/usa
$ ls
A.txt
\end{Verbatim}

\bigline

\item \textbf{\texttt{mv}}
\index{mv@\texttt{mv}}
\index{file!renaming}

\texttt{mv} has pretty much the same effect as \texttt{cp}, except that it
does not retain the original copy. This command can be used to rename a file
(``\texttt{\$ mv oldfilename newfilename}'') as well as to change a file's
location.

\bigline

\item \textbf{\texttt{vim}} (and \texttt{vimtutor})
\index{vim@\texttt{vim}}

It's really ludicrous to include this command in amongst all the others, when
its ins-and-outs could (and do) occupy entire textbooks in their own right.
\texttt{vim} is a text editor program with a zillion amazing features which
you will use this semester to write your programs. The normal way of creating
a file, in fact, will be this:

\begin{Verbatim}[fontsize=\small]
$ vim  notesOnTheResistance.txt
\end{Verbatim}

or this:

\begin{Verbatim}[fontsize=\small]
$ vim  DestroyGalacticRepublic.java
\end{Verbatim}

after which you will do loooooooooooots of other stuff way beyond the scope of
this book. That stuff will be cryptic and agonizing at first, but will
eventually become second-nature and give you the tremendous text editing power
you need to be a truly efficient software developer. It's kind of like
learning to use the Force for the first time.

\index{vimtutor@\texttt{vimtutor}}
For now, I'll make this (strong) suggestion: to learn \texttt{vim} for the
first time, type this command (all one word) at the command line:

\begin{Verbatim}[fontsize=\small]
$ vimtutor
\end{Verbatim}

\index{Coke}
Grab a Coke, and spend 30-40 minutes patiently reading and following the
instructions. This tutorial is quite good, and will teach you the very basics
of getting a file created and edited with this incredible tool.

\bigline

\item \textbf{\texttt{git}}
\label{introduceGit}
\index{git@\texttt{git}}
\index{version control system}

\texttt{git} is another one that doesn't really fit in this list, since it's
much more than just ``a command.'' For now, though, all you need to understand
is that it's a \textbf{version control system} that allows you to track and
manage the changes you make to your software over time.

Up until now, you've been dealing with a paradigm like ``the IDE always has the
most recent copy of my code, and that's the only version of it that exists.''
You'll need much more flexibility than that when you work on large systems.

Here's all you need to know at present, though:

\begin{compactitem}
\index{git@\texttt{git}!\texttt{git init}}
\index{repository@``repo'' (repository)}
\item The command ``\texttt{git init .}'' (don't forget the dot at the end,
after a space) creates a git \textbf{repository}
(or ``\textbf{repo}'') in the current directory. That just means that your
current directory, and everything under it, are now ``under git's
management.''
\index{git@\texttt{git}!\texttt{git add}}
\item You use ``\texttt{git add}'' to make git aware of
one or more files that you want it to track from that point forward. You'll
type ``\texttt{git add }file1 file2 file3'' or however many files you want
to add at that point. Ordinarily you'll want to \texttt{git add} all of your
\texttt{.java} files.
\index{git@\texttt{git}!\texttt{git commit}}
\index{commit}
\index{snapshot}
\item When you've made a
significant change to one or more of your files that you want git to be aware
of, you'll enter this command:
\begin{Verbatim}[fontsize=\footnotesize]
$ git  commit  -a  -m  "A message describing the change."
\end{Verbatim}
Each such change is called a \textbf{commit}. Think of it as taking a
snapshot of your code that you can return to later.
\index{git@\texttt{git}!\texttt{git status}}
\index{git@\texttt{git}!\texttt{git log}}
\item ``\texttt{git status}'' and ``\texttt{git log}'' are two useful
commands that show the current state of your files as git sees them, and a
history of all the different commits you've made. Type them occasionally just
to get a feel for what kind of information they show.
\end{compactitem}

We'll talk much more about \texttt{git} later. For now, just know that it
exists, and type the above commands verbatim when prompted.

\bigline
\item \textbf{\texttt{javac}} and \textbf{\texttt{java}}

\index{javac@\texttt{javac} (compiler)}
\index{java@\texttt{java} (virtual machine)}
\index{compiler}
\index{virtual machine}
\index{JVM (Java Virtual Machine)}
\index{JDK (Java Development Kit)}
\index{J2SE (``Java 2 Standard Edition'')}
\index{JRE (Java Runtime Environment)}

\index{source file}
Now, finally to some programming stuff. On your Linux system, the Java
\textbf{compiler} (\textit{i.e.}, the program that converts your source code
into the form the computer needs to run it) is called \texttt{javac}, and the
\textbf{virtual machine} (the interpreter that runs your compiled code) is
called \texttt{java}. Both of these are part of the \textbf{JDK}, or ``Java
Development Kit,'' that you install in order to program in Java.\footnote{Just
to confuse you, the JDK has sometimes been called the ``Java SDK'' (``Java
Software Development Kit'') and the ``J2SE'' (``Java 2 Standard Edition'') in
the past, and you'll likely run across those acronyms as well. To confuse you
even more, the software you need to simply \textit{run} a Java program (as
opposed to writing your own) is called the ``JRE'' -- Java Runtime
Environment. Finally, to confuse you yet further, Java version numbers were
originally all ``one-dot-something'' (like ``Java 1.3'') but in 2004 they
ditched the ``one-dot'' and started naming the versions after the second
number alone. (So, the successor to ``Java 1.4'' was ``Java 5.'') This book
assumes you're on Java 17 or later, by the way.}


\index{compiler}
To compile, you give \texttt{javac} all the Java files that are part of your
program:

\begin{Verbatim}[fontsize=\scriptsize]
$ javac  DestroyGalacticRepublic.java  Bombs.java  SinisterPlans.java
\end{Verbatim}

\index{java file@\texttt{.java} file}
\index{class file@\texttt{.class} file}
\index{main method@\texttt{main()} method}
which will either produce a \texttt{.class} file for each \texttt{.java} file,
or compiler errors for you to read. Finally, to run it, you give \texttt{java}
the name of \textit{the class that contains your \texttt{main()} method}:

\begin{Verbatim}[fontsize=\footnotesize]
$ java  DestroyGalacticRepublic
\end{Verbatim}

(Notice we don't include ``\texttt{.java}'' or ``\texttt{.class}'' here, and
notice we don't mention every Java class, only the one that has the
\texttt{main()}.)

\end{enumerate}

\pagebreak
\section{The quickest path through the woods}

Whew. That was a lot. It's kind of like moving to another country: every
little thing, all at once, seems different.

All I can do is promise you it will get easier as you get used to that new
country. And there will be parts of it you will like -- maybe you'll even like
it better than the point-and-click country you grew up in.

\index{Hello World@``Hello, World"!'' program}
In the meantime, let's pull together all the steps to get a ``Hello World''
Java program running on the Linux command line.

\begin{enumerate}
\itemsep.1em
\item Log on to your Linux system (for instance, the UMW server or your Google
Cloud instance), however you do that.
\item Create a directory to hold your project:
\begin{Verbatim}[fontsize=\small]
$ mkdir  myFirstProgram
\end{Verbatim}

\item And make sure to actually go there:
\begin{Verbatim}[fontsize=\small]
$ cd  myFirstProgram
\end{Verbatim}

\item Create a git repo to manage this project:
\begin{Verbatim}[fontsize=\small]
$ git  init  .
\end{Verbatim}

(and of course don't forget that pesky dot at the end.)

\item Now create a Java file:
\begin{Verbatim}[fontsize=\small]
$ vim  HelloWorld.java
\end{Verbatim}

\textbf{(You are now in \texttt{vim}. Everything you learned during your
\texttt{vimtutor} session, and everything you can get from a zillion different
``vim cheat sheets'' on the Internet, is relevant now. Good luck.)}

\item Give it these contents:
\begin{samepage}
\begin{Verbatim}[fontsize=\small,frame=single]
class HelloWorld {
    public static void main(String args[]) {
        System.out.println("yo sup dawg");
    }
}
\end{Verbatim}
\end{samepage}

\item Save your file and exit \texttt{vim}.

\item Now compile it:
\begin{Verbatim}[fontsize=\small]
$ javac  HelloWorld.java
\end{Verbatim}

\item And, since it gave you no errors, run it:
\begin{Verbatim}[fontsize=\small]
$ java  HelloWorld
\end{Verbatim}

\item Finally, add the file to your repo:
\begin{Verbatim}[fontsize=\small]
$ git  add  HelloWorld.java
\end{Verbatim}

\item and commit it:
\begin{Verbatim}[fontsize=\small]
$ git  commit  -a  -m  "Finished chapter 1!"
\end{Verbatim}

\end{enumerate}

It's a big bright world ahead of us. Go take a break and I'll see you next
chapter.


\chapter{The ``cells" of an OO program}
\label{ch:cells}

Java is called an ``object-oriented" programming language. If I were King of
the World, I would have called it a ``\textit{class}-oriented" language
instead. That's because in Java, you don't write code for objects, but for
\textit{classes}, and that code then defines the behavior of the classes that
are based on them.\footnote{There are other languages, for instance JavaScript
(no relation to Java), which do deserve the term ``\textit{object}-oriented,"
since you can create code for individual objects rather than classes, and not
every object has to have a class at all.} You'll sometimes hear people
mistakenly say stuff like, ``I wrote some code for the DatabaseConnection
object today." It makes me wince. They weren't writing ``code for the
object," but the code for a class.

If an OO program is an organic entity, classes and objects comprise its cells.

\section{Terms}

So here's a crucial pair of definitions. A \textbf{class} is a
\textit{category} of things. An \textbf{object} is a concrete \textit{example}
of a class. If ``University" is a class, then ``UMW" is an object; if
``Course" is a class, then ``CPSC 240" is an object. The difference is real,
and it is vitally important to keep at the forefront of your mind as you begin
your OO quest. Getting them mixed up is like Peter Venkman crossing the
streams.

You'll sometimes hear alternate definitions of these terms, like ``a class is
a template, and objects are copies of that template." This is better than
out-and-out confusion, but it still misses something important. It's an
operational definition, instead of a conceptual definition. It thinks of a
class and an object in terms of the mechanical way the virtual machine carries
out their duties, rather than in terms of \textit{modeling}, which is what
OOA\&D is all about.

In our world, every single software object will be a member of a category,
and that category will define everything about its inner structure and rules
of behavior.

By the way, an important near-synonym for class is \textbf{type}. (It's only a
\textit{near}-synonym because primitive, non-classes like \texttt{int}s and
\texttt{boolean}s are also types.) An important exact synonym for object is
\textbf{instance}.

In addition to those nouns, a big verb in our vocabulary will be the term
\textbf{instantiate}. It means ``to actually create an object of a particular
class." Some people use words like \textbf{construct} or \textbf{create} for
this, or even ``\texttt{new}" (or ``\texttt{new} up") as a verb, but for the
most part we'll stick with instantiate.


\section{A different kind of language}
\label{sec:UMLclasses}

Classes and objects are among the basic building blocks of any OO program, and
they will play a prominent role on various \textbf{UML diagrams}. UML
(``Unified Modeling Language") is a \textit{design} language, not a
programming language. It is expressed in visual diagrams, not streams of text.
Even though it's not text-based, though, and even though there's no
``compiler" forcing us to adhere to the syntax, it still has rules that must
be followed, and precise meanings that can be inferred.

Figure~\ref{fig:classObject} shows what a class, and an object, look like in
UML. (I'm putting classes in yellow and objects in blue, but those colors
aren't part of UML itself, just the black-and-white stuff.) Both are boxes,
but notice the class box has three compartments in it while the object box has
two.

\begin{figure}[ht]
\centering
\includegraphics[width=1.0\textwidth]{classObject.pdf}   % 650x350
\caption{A class (left) and an object in UML.}
\label{fig:classObject}
\end{figure}

\subsection{Classes in UML: the first two compartments}

Let's look at the class in detail. In the top box is its name; so far so good.
One thing to point out, though, is that in Java, \textit{the names of all
classes are capitalized.} Don't ever violate this rule, for convention's and
confusion's sake!

The second compartment has the class's \textbf{instance variables}. You'll
hear people use other terms for these like like ``member variables" and even
``class variables," but I strongly prefer instance variables (or ``inst vars"
for short) and here's why: \textit{every instance has its own copy of an
instance variable.} This truth is absolutely fundamental to OOP, and it's
worth re-reading that sentence again and again until it's part of your core
being. Declaring a plain-ol' variable like ``\texttt{int x;}" creates a single
storage location in which a value can be stored. But declaring an
\textit{instance} variable is a far-reaching choice that destines every Car
(or whatever) that will come about in the future to have its own copy of that
variable. It's our way of defining the very structure of Cars in perpetuity.

One slight headache is that the UML syntax differs from Java's a bit: instead
of listing the variable's type and then its name, we reverse them, we use
a colon instead of a space, and we omit the semicolon. Otherwise, it's pretty
straightforward to interpret that second compartment.

By the way, one important piece of syntax in that second compartment is an
\underline{underline}, which says that the underlined ``instance variable"
isn't an instance variable after all: it's a \textbf{class variable}. This
means that \textit{there's only one shared variable for the entire class,
rather than a different variable for each object.} In
Figure~\ref{fig:classObject}, the integer \texttt{numCars} variable is such a
case. Even though every \texttt{Car} has its own \texttt{make},
\texttt{model}, \texttt{odo}meter reading, \textit{etc.}, they all share one
\texttt{numCars} (which presumably represents the total number of \texttt{Car}
objects instantiated so far.) This makes sense, since after all such a
variable is not specific to a certain \texttt{Car}. We'll see that in Java,
class variables are created by using the ``\texttt{static}" keyword where the
variable is declared.

\subsection{Classes in UML: the third compartment}

The third compartment isn't much harder: it contains the \textbf{methods} for
the class. Like everything it seems, programmers have multiple terms for these
two: they're called \textbf{member functions} or \textbf{class functions} on
occasion. We'll stick with \textbf{method}.

The crucial distinction between a method and a regular ol' Joe function is
this: while you can call a function to trigger it, you must call a method
\textit{on an object}. In the example, we have a \texttt{fillUp()} method
defined on the \texttt{Car} class. Since it's not an ordinary function, but
rather an OO method, we must call it on a particular instance of a
\texttt{Car}. In Java code, this does \textit{not} work:

\begin{verbatim}
    fillUp();        // NOPE
\end{verbatim}

nor does this:

\begin{verbatim}
    Car.fillUp();    // NOPE
\end{verbatim}

Instead, one must call \texttt{fillUp()} like this:

\begin{verbatim}
    johnsMercedes.fillUp();     // Correct!
\end{verbatim}

where \texttt{johnsMercedes} is the name of a valid \texttt{Car} object,
previously instantiated.

Beginners sometimes view this as a syntactic nuisance. It is not. It is
fundamental to what your code \textit{means}. Conceptually, it makes sense to
have a particular car, and to fill it up. It does \textit{not} make sense to
say ``hey universe, fill up cars" (which is what ``\texttt{fillUp()}" seems to
say) not to say ``hey Cars-in-general, fill yourself up" (which is what
``\texttt{Car.fillUp()}" seems to say).

By the way, notice in the example I just gave, \texttt{johnsMercedes} is
\textit{not} capitalized. (The capital ``\texttt{M}" in the middle doesn't
count; that's just an artifact of CamelCase, which is a way of making multiple
words easier to read.) This is always the case: in Java, object names always
begin with a lower-case letter.

Back to the third compartment. You can probably tell that the stuff inside the
parentheses is arguments to the respective methods, with the same
name-first-then-type colon-syntax, and you can probably tell that after the
closing parenthesis, you have the return type of the function. All of this
looks vaguely Java-like, and that's because even though a UML diagram is
technically programming-language-independent, language-specific things like
\texttt{int} and \texttt{String} can't help but creep in in practice. Our
thoughts betray us.

\subsubsection{Various ``special" methods}

A few of those methods are worthy of special note. The first one listed,
called simply ``\texttt{Car}", is a very special kind of method called a
\textbf{constructor} which we'll be talking about a lot. Here's an iron-clad
rule which is fundamental to much that follows: \textit{whenever an object is
instantiated, one of its class's constructors is called.} This happens
automatically; it's not something we have to do ourselves. (Java's syntax for
this, as we'll see, makes it kind of look like we're calling the constructor
ourselves, which is a mixed blessing.) In Java, there are two things that
``make" a method a constructor: (1) it must have exactly the same name as the
class, and (2) it must have \textit{no} return type. (Note that ``no" return
type is not the same as a \texttt{void} return type! I mean \textit{no return
type at all.}\footnote{If you mistakenly include a \texttt{void} before your
constructor when you write the code, you have officially made it \textit{not}
a constructor anymore! It's now just an ordinary method -- weirdly named the
same as the name of the class it's in -- which will not be automatically
invoked at instantiation time as a constructor should. I once had a nasty bug
at the eleventh hour of a software release because of this exact issue.})

By the way, just as a class can have multiple methods with the same name as
long as those methods have different argument lists, so it can have multiple
constructors subject to the same conditions. This is a very common practice,
although in this first example we have only one \texttt{Car} constructor.

Also, just as in the second compartment, an \underline{underline} indicates
that the method ``goes with the whole class, not with each object." And just
as before, this implies the use of the \texttt{static} keyword. A
\texttt{static} method is essentially a function: \textit{i.e.}, you
\textit{don't} call it on an object. Instead, you just call it \textit{on the
class itself.} In the example above, \texttt{numCars()} method is
\texttt{static}, which means that you could write ``\texttt{Cars.numCars()}"
to retrieve the number of \texttt{Car} objects that have been instantiated to
that point. Static methods are quite rare, but they do arise occasionally, and
are always indicated with an underline.

The other methods I'll draw your attention to are the ones that begin with
``\texttt{get}". People call these methods ``\textbf{getters}," and all they
normally do is return the value of the instance variable in question. Often
one also has ``\texttt{set}" methods to set the values of inst vars, although
our example doesn't have any of those. People also call getters and setters
\textbf{accessors}, and sometimes specifically call setters
``\textbf{mutators}," a term which always made me chuckle.

\subsection{Objects in UML}

The blue object in the diagram has only two compartments, not three. That's
because there's no need (in most OO languages) to say anything about an
object's methods when focusing on the object: after all, the methods are
simply defined by the class, and are common to all instances of that class. It
is important, however, to specify the current \textbf{state} of the object,
which means the current values of all its instance variables. In the picture,
you can see that there is a \texttt{Car} object in memory representing an old
Chevy Malibu with a zillion miles on it and other suboptimal features.

Perhaps the strangest thing about a UML object is the top compartment. Notice
that it says ``\texttt{:~Car}" (``colon-Car"), which is not a typo. Here's the
sitch. The top compartment of a class has the class's name, since that's all
there is to say about it. The top compartment of an object, meanwhile, has the
\textit{object's} name, followed by a colon and then its class. Just like we
said ``\texttt{make :~String}" earlier, so here we can say
``\texttt{johnsMercedes :~Car}". Why, then, is Figure~\ref{fig:classObject}
missing the name before the colon? Because we've chosen \textit{not to name
this object in the diagram.} It's just ``a Car" with certain properties, not a
named Car. This may seem odd, but in fact 99\% of the time we will do exactly
this. And that's because bizarrely, \textit{objects don't have names in Java},
even though it may seem at first that they do.

More on that later. For now, just accept the fact that UML diagrams can depict
objects, and normally we don't choose to specify the object's name -- only its
type and its instance variable values.


\subsection{The value of ``design"}

Before we move on to implementation, take a step back for a moment and
consider the \textit{information} contained in that
Figure~\ref{fig:classObject}.

Suppose you were given the job to write a car maintenance tracking program,
and you were getting started figuring out how to accomplish that. I think
you'll agree that if someone handed you that diagram, it would be valuable
indeed. There's no code in it \textit{per se}, but a great deal of the work
has already been done for you! You already know what to name your class, the
names and types of all its constituent variables, and what methods its objects
should support. With that diagram alone, I'd say 70\% of the work has been
done. All it takes now is to convert that diagram into Java (or whatever
language you're working with), and flesh out the methods to do the right
thing. But the overall blueprint communicates a ton of information about
decisions that have already been made. Your structure is defined, and now you
just need to bust out a hammer and some nails.


\section{Classes in Java}

In Java, every class is in its own file\footnote{Technically there can be
some exceptions to this, but don't worry about them now.} named the same as
the name of the class (including the capital letter) with a \texttt{.java}
extension. Operationally, we can use \texttt{vim} to create it and edit it:

\begin{verbatim}
$ vim Car.java
\end{verbatim}

The skeleton of any class file -- after the \texttt{package} and
\texttt{import} statements we'll talk about later -- is the class definition,
with curly braces:

\begin{Verbatim}[samepage=true,fontsize=\footnotesize,frame=single]
class Car {
    
}
\end{Verbatim}

You may be used to putting the word ``\texttt{public}" before the word
\texttt{class} here. For now, we won't do this, and I'll encourage you to
ditch the habit of making classes \texttt{public} by knee-jerk reaction. As
we'll learn, you want to lean towards making things ``as private as possible"
until you have reason to do otherwise.

\subsection{Instance variables in Java}

Instance variables go directly inside the class definition, and outside of any
method:

\begin{Verbatim}[samepage=true,fontsize=\footnotesize,frame=single]
class Car {
    String make, model;
    int yearsOld;
    int odo;
    double galsRemaining;
    double sizeOfTank;
    double gasMileage;    
}
\end{Verbatim}

You may be used to seeing the word ``\texttt{private}" before each instance
variable, and I do applaud that practice. More on that later. For now, we'll
leave it off just because it's not necessary to compile. Realize that it's not
the word \texttt{private} that makes something an instance variable; rather,
it's the fact that it's defined directly inside the class, rather than within
a method.

\subsection{Constructors in Java}

Next on the diagram is our constructor. We put in the boilerplate to get us
started:

\begin{Verbatim}[samepage=true,fontsize=\footnotesize,frame=single]
class Car {
    ...

    Car(String make, String model) {

    }
}
\end{Verbatim}

and now for the first time we have to actually \textit{think}.

A constructor, as I said, is automatically called whenever an object comes
into existence. This is our ``hook" to set up the object for success when
methods are called on it later. Think of it this way: your constructor is
called whenever a new object is about to come off the assembly line and enter
the real world. Your job in the constructor is to do everything necessary to
make sure it's ready for prime time.

Often this will involve initializing the instance variables to reasonable
values. Sometimes it will include other things, like registering its existence
in some global repository of objects, or initializing a connection to a
network, or writing itself to a database. The key question to ask yourself is,
``what do I need to do to ensure this object is `legit' and doesn't break
anything when it's being used?"

In our case, the instance variables are all that matters. First, let's set the
object's \texttt{make} and \texttt{model} to what was given to the
constructor:

\begin{Verbatim}[samepage=true,fontsize=\footnotesize,frame=single]
class Car {
    ...

    Car(String make, String model) {
        this.make = make;
        this.model = model;
    }
}
\end{Verbatim}
\normalsize

If this is the first time you've seen the odd word ``\texttt{this}" in a
program, have a good chuckle. What a weird word choice! But Gosling \& Co.
chose this word to denote a central OO programming concept. The word
``\texttt{this}" means one of two different things, and they both need to be
memorized:

\begin{enumerate}
\large
\itemsep.1em
\item Inside a \textit{constructor}, ``\texttt{this}" means ``the object that is
currently being instantiated."
\item Inside a \textit{method}, ``\texttt{this}" means ``the object the
method was called on."
\item (Anywhere else, ``\texttt{this}" is illegal.)
\normalsize
\end{enumerate}

It's weird and meta and self-referential, but it's also necessary. There are
times when we need to have a name for ``the very object I'm 'in' right now,"
and ``\texttt{this}" is our (awkward) name to refer to that.

So in our constructor, when we say ``\texttt{this.make}" we mean ``the
\texttt{make} instance variable of this very object that is in the process of
being birthed. We set that to the \texttt{make} argument that was passed to
the constructor. Ditto with \texttt{model}. Oftentimes, using \texttt{this} is
optional, but in this case it's required because we named our argument the
same as the instance variable, and there has to be a way to distinguish
between the two.

Now for our other inst vars. Some of them make sense to be set to zero:

\begin{Verbatim}[samepage=true,fontsize=\footnotesize,frame=single]
class Car {
    ...
    Car(String make, String model) {
        this.make = make;
        this.model = model;
        this.yearsOld = 0;
        this.odo = 0;
        this.galsRemaining = 0.0;
    }
}
\end{Verbatim}

since brand new cars are in fact zero years old, have a 000000 odometer, and
have no gas in their tank (maybe). The other two don't, however; brand new
cars still have a gas tank of a certain size, and they certainly get more than
0 mpg. For this example, I'm going to go with a very limited notion of
automotive properties:

\begin{Verbatim}[samepage=true,fontsize=\footnotesize,frame=single]
class Car {
    ...
    Car(String make, String model) {
        this.make = make;
        this.model = model;
        this.yearsOld = 0;
        this.odo = 0;
        this.galsRemaining = 0.0;
        
        if (make.equals("Chevy") || make.equals("GM")) {
            sizeOfTank = 21;
        } else {
            sizeOfTank = 13;
        }
        if (make.equals("Chevy") && model.equals("Malibu")) {
            gasMileage = 3;
        } else {
            gasMileage = 24;
        }
    }
}
\end{Verbatim}

I'm totally not bitter about my car's gas mileage, by the way.

\subsection{Methods in Java}

The other methods follow a similar syntactic pattern. But it's super important
to keep this truth in the front of your mind: \textit{because they are methods
(not functions), they are called \underline{on an object.}} That means that
you can refer to instance variables inside of them -- when you do, you're
talking about \textit{the instance variables of the object the method was
called on.} Put another way, you're talking about the instance variables of
\texttt{this}.

\subsubsection{Thinking reactively}

When you write methods in an OO program, you have to think reactively, not
proactively. What I mean is this. When you write a procedural, old school
program, you're the one driving the ship. In your \texttt{main()} you say,
``first do this, then do that; create these three variables, perform a
computation, and then print the result." You're the one in charge of the
story, and you spell out how it's going to go in order.

We all learned how to program this way. But in OO, you kind of have to think
backwards from that. Writing a method isn't like calling it; instead of giving
orders, you're providing a service to whoever called you. So when you write a
method, you have to think, ``okay, some other part of the code is now calling
me, for reasons of its own. What do I do in response to that?"

Often that involves updating the object's \textbf{state} to reflect what
should happen to it as a result of the method being called.

This is best seen by example. Let's implement\footnote{If you don't know this
term, the verb \textbf{to implement} means ``to take a design and actually
build it out." It is a synonym for the verb \textbf{to code}.} the
\texttt{.fillUp()} method first. Don't think about Java; think about cars. If
I fill up a car, what happens?

Does the make or model or mileage change? Of course not: the amount of gas in
the tank does. And ``fill 'er up" means to raise it to its maximum. The
correct implementation of \texttt{.fillUp()}, then, is simply:

\begin{Verbatim}[samepage=true,fontsize=\scriptsize,frame=single]
class Car {
    ...
    void fillUp() {
        galsRemaining = sizeOfTank;
    }
}
\end{Verbatim}

We could equally well have written this as:

\begin{Verbatim}[samepage=true,fontsize=\scriptsize,frame=single]
class Car {
    ...
    void fillUp() {
        this.galsRemaining = this.sizeOfTank;
    }
}
\end{Verbatim}

to make it explicit that we're talking about two instance variables here, and
assigning the value of one to the other. It's a matter of style.

In the same vein, we ask ourselves, ``suppose someone asks me what percentage
full my tank is. What answer do I give them?" The proper response again
involves the same two inst vars and a little math:

\begin{Verbatim}[samepage=true,fontsize=\scriptsize,frame=single]
class Car {
    ...
    double getTankPerc() {
        return galsRemaining / sizeOfTank * 100;
    }
}
\end{Verbatim}

I chose to omit \texttt{this}, but again it's a personal choice.

Some methods are no-brainers:

\begin{Verbatim}[samepage=true,fontsize=\scriptsize,frame=single]
class Car {
    ...
    String getModel() {
        return model;
    }
}
\end{Verbatim}

If someone asks me what my model is, I tell them my model, duh. The same is
true for the other accessor methods.

Finally, what if someone drives me $n$ miles? How should my internal state
be adjusted to reflect that?

This is the most difficult one, and again it requires you to think about cars
rather than about Java. Mentally run through the variables we've chosen to
represent a car, and ask yourself which ones need to change, and how? You'll
realize that the odometer and the gas tank level are the two we need to
modify. When someone drives a car $n$ miles, the odometer needs to increase by
$n$ miles (else it ain't legal); also, the gas tank needs to be reduced by
$\frac{n}{m}$ gallons, where $m$ is the car's gas mileage in mpg. So here we
go:

\begin{Verbatim}[samepage=true,fontsize=\footnotesize,frame=single]
class Car {
    ...
    void drive(int numMiles) {
        double galsBurned = numMiles / this.gasMileage;
        this.galsRemaining = this.galsRemaining - galsBurned;
        this.odo += numMiles;
    }
}
\end{Verbatim}

This time, I did include the \texttt{this}'s where appropriate, since we also
have a couple of local variables involved and I wanted to be explicit. Our
math is a mix of function parameters, temporary variables, and permanent
attributes of the \texttt{Car}.

\subsubsection{There's always an \texttt{Exception...}}

The shrewd reader (and driver!) will realize that our \texttt{.drive()} method
is a bit optimistic: when told to drive \texttt{numMiles}, it blindly does so,
even if there's not enough gas to get that far. We ought to guard against this
kind of wishful thinking by \textit{not permitting} a drive that's outside our
range. If told to drive 1000 miles when we only have enough gas to go 200,
we'll just say no. That's way better than ending up with a negative gas tank
and wreaking unknown havoc later in the program!

The first step in implementing this kind of defensive coding is to figure out
\textit{when} to refuse to carry out orders. That's not too hard in this case:
our local \texttt{galsBurned} variable is exactly what we need: if it turns
out to be higher than the gas remaining in the tank, we are officially in
Nonsense Land. A simple \texttt{if} statement can take care of that.

The second step is figuring out \textit{what to do} when this occurs. Most
people's first thought is to blare out a siren:

\begin{Verbatim}[samepage=true,fontsize=\scriptsize,frame=single]
// Inadequate approach #1
class Car {
    ...
    void drive(int numMiles) {
        double galsBurned = numMiles / this.gasMileage;
        if (galsBurned > this.galsRemaining) {
            System.out.println("Not enough gas!!");
        }
        this.galsRemaining = this.galsRemaining - galsBurned;
        this.odo += numMiles;
    }
}
\end{Verbatim}

This is done in the hopes that someone will hear us and be alarmed. The
problem is, this error will go to the console, where it may or may not ever be
seen; and even if someone notices it, we've still already done the dirty deed.
We have a \texttt{Car} object with an \textbf{illegal state}: a negative gas
level.

Slightly better, but still not good enough, is to print the error and also
refuse to carry out orders:

\begin{Verbatim}[samepage=true,fontsize=\scriptsize,frame=single]
// Inadequate approach #2
class Car {
    ...
    void drive(int numMiles) {
        double galsBurned = numMiles / this.gasMileage;
        if (galsBurned > this.galsRemaining) {
            System.out.println("Not enough gas!!");
            return;                                     //  <---  NOTICE THIS!
        }
        this.galsRemaining = this.galsRemaining - galsBurned;
        this.odo += numMiles;
    }
}
\end{Verbatim}

Now, in addition to printing the error, we also \texttt{return} from the
method prematurely, before carrying out the nonsensical operation.

The problem with this approach is that \textit{the calling code is not alerted
that anything went wrong.} We'll see some ``calling code" in action in the
next section, but for now just realize that whoever called
\textit{drive(1000)} is none the wiser. It merrily chugs along thinking that
the thousand-mile drive was plain sailing, oblivious to the fact that no such
drive actually occurred.

The right way to handle this is with Java's \texttt{Exception} mechanism. We
don't return prematurely, as above; rather, \textit{we don't return at all.}
Exceptions are Java's way of allowing a method \textit{not} to return, but
instead to raise a big red flag that carrying out the method just flat didn't
work. It's the only responsible thing to do.

Our first step is to ``throw an exception" instead of returning. Here's the
code to do so:

\begin{Verbatim}[samepage=true,fontsize=\scriptsize,frame=single]
// Correct approach
class Car {
    ...
    void drive(int numMiles) {
        double galsBurned = numMiles / this.gasMileage;
        if (galsBurned > this.galsRemaining) {
            throw new Exception("Not enough gas!!");    //  <---  NOTICE THIS!
        }
        this.galsRemaining = this.galsRemaining - galsBurned;
        this.odo += numMiles;
    }
}
\end{Verbatim}

Operationally, throwing an exception has the same immediate effect as
returning: the method instantly terminates and returns control back to the
caller. The difference, as we'll see in the next section, is that the caller
is aware of the difference, does not get a return value, and can take evasive
action.

If you try to compile the above code, though, you get an error, which says:

\begin{Verbatim}[samepage=true,fontsize=\small]
error: unreported exception Exception; must be caught or declared to be thrown
\end{Verbatim}

This is good of Java, and here's why. Inside our method, we've created a
possibility that we won't return at \textit{all}, and will instead barf with
this error. But Java requires that if we do that, we 'fess up and declare that
this is a possibility. That prevents unwitting programmers from blithely
calling our method and not accounting for the fact that it might not even
work.

Fixing it is simple; we just change the first line of the function to:

\begin{Verbatim}[samepage=true,fontsize=\scriptsize,frame=single]
    void drive(int numMiles) throws Exception {
        ...
\end{Verbatim}

We're saying, ``you can call me on a \texttt{Car}, pass me an integer
argument, and get no return value. BUT there's a possibility that it won't
work, and you should be aware of that." It's only honest, and as we'll see, it
allows the code that uses \texttt{Car}s to properly deal with the problem.



\section{Objects in Java}

We've now coded a class from the ground up (the complete code listing is in
Figure~\ref{fig:carClassCode}.) The reason we did this was so we can
instantiate objects of that type and do something with them. Let's make a
simple \texttt{main()} method to do just that.

You'd be surprised how many beginning programmers try to drive 23 miles like
this:

\begin{Verbatim}[samepage=true,fontsize=\scriptsize,frame=single]
    public static void main(String args[]) {
        drive(23);    // WRONG!
    }
\end{Verbatim}

or this:

\begin{Verbatim}[samepage=true,fontsize=\scriptsize,frame=single]
    public static void main(String args[]) {
        Car.drive(23);    // equally WRONG!
    }
\end{Verbatim}

You'll get compiler errors, but those errors reflect a deeper and more
fundamental misunderstanding. In OOP, you have to call a method \textit{on an
object}. Conceptually, nothing else makes sense. In real life you don't
``drive in general," and you don't ask ``automobiles in general" to drive you
places. Instead, you have to drive \textit{a particular car} somewhere. Here's
how:

\begin{Verbatim}[samepage=true,fontsize=\scriptsize,frame=single]
    public static void main(String args[]) {
        Car minivan = new Car("Toyota","Sienna");
        minivan.drive(23);    // correct!
    }
\end{Verbatim}

The keyword ``\texttt{new}" is utterly crucial here. In Java, \textit{the only
way to instantiate an object is to use \texttt{new}}. It causes a fresh object
of the appropriate type to spring into existence, complete with memory to hold
its instance variables. And the appropriate constructor is called, of course,
to set that object up for prime time.

We got errors before because we didn't even \textit{have} a car to do anything
with. There was no memory set aside, no constructor called to set up the
object, nothing. We tried a shortcut, and that was madness. But now that we
know how to instantiate objects, we can do so to several and create a whole
new world:

\begin{Verbatim}[samepage=true,fontsize=\footnotesize,frame=single]
    public static void main(String args[]) {

        // The archaic Davies family vehicles
        Car minivan = new Car("Toyota","Sienna");
        minivan.setYear(2002);
        Car stephensLemon = new Car("Chevy","Malibu");
        minivan.setYear(2001);

        // Grammy lives in Colorado
        Car grammysCar = new Car("Lexus","ES");
        grammysCar.setYear(2018);

        // Caravan to Disneyworld -- whoo-hoo!  (Grammy's meeting us there.)
        minivan.fillUp();
        minivan.drive(500);
        stephensLemon.fillUp();
        stephensLemon.drive(500);
        System.out.println("The van is " + minivan.getTankPerc() + 
            "% full, while the chevy is " + stephensLemon.getTankPerc() +
            "% full.");
        grammysCar.drive(1899);  // a long way from Colorado
    }
\end{Verbatim}

All this code is legit, and shows that our \texttt{Car} class has uses. We can
represent each car's attributes in our caravan, keep track of how much gas it
has and when it needs to be refilled, \textit{etc.}

The only fly in our ointment (don't worry; he's easily swatted) is the
aforementioned \texttt{.drive()} method, and how it might not always run to
completion. In fact, if we compile the above main program, we'll get the same
kind of compile error that we did when we were midway through implementing the
Exception-throwing stuff. It'll say we're being unconscionably remiss by
refusing to deal with the error that might occur every time we tell one of our
cars to drive.

The Java way to handle this is with a \textbf{try/catch block}. Essentially,
this just builds a little scaffolding around our call to suspicious methods
like \texttt{.drive()} so that if and exception is indeed ``thrown" when we
call it, we can ``catch" is and do something sensible. Here's how:

\begin{Verbatim}[samepage=true,fontsize=\scriptsize,frame=single]
    public static void main(String args[]) {
        ...
        // Caravan to Disneyworld -- whoo-hoo!  (Grammy's meeting us there.)
        minivan.fillUp();
        try {
            minivan.drive(500);
        } catch (Exception e) {
            System.out.println(e);
            System.exit(1);
        }
        ...
    }
\end{Verbatim}

Instead of simply calling \texttt{minivan.drive()} and throwing caution to the
wind, we put that code in a \textbf{try block}. The code in a \texttt{try}
block is executed normally, step-by-step, just like anywhere else in a Java
program. But the \texttt{try} block has one or more (here just one)
contingency plans connected to the bottom of it, which can handle any special
(or ``exceptional") conditions. In this case, that call to drive the minivan
500 miles through traffic will either work in its entirety and have the
desired effect, or it will abort in the middle of it and control will be
immediately transferred to the relevant catch block below it. The flow
continues in that \textbf{catch block}, in this case printing the message of
the \texttt{Exception} and then terminating the program.

What to do in each exceptional situation depends on the situation itself.
Sometimes, there are meaningful things one can do in response to an error:
like if a network connection fails, the code can retry connecting; or if a
checking account withdrawal fails, the system can cut over to the savings
account and cover the amount from those funds instead. In our case, if our
program tracked things like routes and desired destinations, a caught
exception would indicate that we need to find a new, temporary destination
other than the one we're currently seeking: one that has gas so we can fill
up.

Once all the method calls that are defined as ``\texttt{throws Exception}"
have been enclosed in try/catch blocks that at least nominally handle the
errors, the code compiles again and it can hopefully run error-free.

\subsection{Printing an object}

One last thing before we bring this chapter to a close. Suppose we're
debugging our program, and we want to print out the values of various
variables to help us hunt down an error. Printing an \texttt{int} or other
standard type is straightforward:

\begin{Verbatim}[samepage=true,fontsize=\scriptsize,frame=single]
    int numEnchiladas = 3;
    System.out.println("The number of enchiladas is: " + numEnchiladas + ".");
\end{Verbatim}

and will produce a message like ``\texttt{The number of enchiladas is 3.}"
What happens, though, if we print out an \textit{object}, like a \texttt{Car}?
How can such a complex entity be reduced to a string of text?

Heck, let's try it:

\begin{Verbatim}[samepage=true,fontsize=\scriptsize,frame=single]
    Car porsche = new Car("Porsche","Carrera");
    porsche.setYearsOld(2);
    System.out.println("My car is: " + porsche + ".");
\end{Verbatim}

The output we get is:

\begin{verbatim}
My car is: Car@4aa298b7.
\end{verbatim}

Whoa. The word ``\texttt{Car}" is perhaps not surprising, but what's the rest
of that gunk?

It turns out that Java's default way of rendering an object as a
\texttt{String} is to concatenate the name of the class, an ``at" sign,
followed by \textit{the memory address} in which it is stored. We'll talk much
more about memory in the next chapter. For now, just think of the memory
address as a unique number\footnote{Yes, it is indeed a number, despite the
fact that it has letters in it like '\texttt{a}' and '\texttt{b}'. It's
printed here in \textbf{hexadecimal}, which is a base-16 number system instead
of the base-10 system non-computer-science humans use.} that identifies the
object, like an SSN.

\begin{samepage}
The cool thing is, we can \textbf{override} this functionality at will, and
change the way \texttt{Car}s will be printed. Check it out. Create a method in
the \texttt{Car} class called \texttt{.toString()}. It must:

\begin{compactenum}
\itemsep.1em
\item be called exactly ``\texttt{.toString()}".
\item take no argument.
\item return a \texttt{String}.
\item have the word \texttt{public} immediately before the return type. (We'll
talk a lot about what \texttt{public} means in future chapters. For now, it just has to
be there.)
\end{compactenum}
\end{samepage}

Here's one:

\begin{Verbatim}[samepage=true,fontsize=\scriptsize,frame=single]
    public String toString() {
        return "a " + yearsOld + "-year-old " + make + " " + model;
    }
\end{Verbatim}

We're assembling various aspects of the vehicle into a sensible, readable
string. Now, when we run \textit{the same code} as above, our output is this:

\begin{verbatim}
My car is: a 2-year-old Porsche Carrera.
\end{verbatim}

Notice that we didn't explicitly ever call \texttt{.toString()}! Instead, we
just used a \texttt{Car} object in a context in which a \texttt{String} was
required, and Java faithfully called our method instead of the one that
generated the memory address. Pretty cool.

This is actually our first foray into a really deep and powerful technique
called ``inheritance," about which much more will come in later chapters. For
now, just grasp the idea that Java lets us override its general behavior for
specific kinds of objects, which gives us tremendous power and flexibility.

\begin{figure}
\begin{Verbatim}[fontsize=\scriptsize,frame=single]
class Car {
    String make, model;
    int yearsOld, odo;
    double galsRemaining, sizeOfTank, gasMileage;

    Car(String make, String model) {
        this.make = make;
        this.model = model;
        yearsOld = 0;
        odo = 0;
        galsRemaining = 0;
        if (make.equals("Chevy") || make.equals("GM")) {
            sizeOfTank = 21;
        } else {
            sizeOfTank = 13;
        }
        if (make.equals("Chevy") && model.equals("Malibu")) {
            gasMileage = 3;
        } else {
            gasMileage = 24;
        }
    }

    public String toString() {
        return "a " + yearsOld + "-year-old " + make + " " + model;
    }

    String getMake() { return make; }

    String getModel() { return model; }

    int getYearsOld() { return yearsOld; }

    void setYearsOld(int x) { yearsOld = x; }

    void fillUp() {
        this.galsRemaining = this.sizeOfTank;
    }

    double getTankPerc() {
        double perc = galsRemaining / sizeOfTank * 100;
        return perc;
    }

    void drive(int numMiles) throws Exception {
        double galsBurned = numMiles / gasMileage;
        if (galsBurned > galsRemaining) {
            throw new Exception("Not enough gas!");
        }
        galsRemaining = galsRemaining - galsBurned;
        odo += numMiles;
    }
}
\end{Verbatim}
\caption{A complete Java implementation of the Car class.}
\label{fig:carClassCode}
\end{figure}



%\section{Exercises}

%
%Use an index card or a piece of paper folded lengthwise, and cover up the
%right-hand column of the exercises below. Read each exercise in the
%left-hand column, answer it in your mind, then slide the index card down to
%reveal the answer and see if you're right! For every exercise you missed,
%figure out why you missed it before moving on.
%
%\begin{small}
%\begin{enumerate}
%\newcolumntype{Q}{>{\arraybackslash}m{.3\textwidth}}
%\newcolumntype{A}{>{\arraybackslash}m{.6\textwidth}}
%%\begin{longtable}{m{0.3\textwidth} || m{0.6\textwidth}}
%\begin{longtable}{Q || A}
%\hline
%\item 
%Would \texttt{Tweet} be a good name for a class, an object, or neither one?
%&
%A class.
%\\
%\hline
%\item
%How about \texttt{nextMessage}?
%&
%An object.
%\\
%\hline
%\item
%How about \texttt{Destroy}?
%&
%Neither one.
%\\
%\hline
%
%\end{longtable}
%\end{enumerate}
%\end{small}


\chapter{Memory matters}

This chapter is near and dear to my heart. The concepts here are vastly
undertaught by computer science educators today, and yet they are at the
epicenter of most intermediate students' understanding (or misunderstanding).
A failure to master this material slaps a hard ceiling on what you can
accomplish as a programmer. Successfully mastering it is the key to the next
level.

The key idea is that there are two ways of looking at a computer program. One
is to look at the static lines of code as they are written on a screen or on
paper. This is how novices think about programs: they look at the lines of
code, and ask themselves whether lines need to be added, removed, changed, or
moved.

The other way is to think about what happens to the computer's \textbf{memory}
as the program runs, and how its variables and structure change as the program
unfolds. Whether they realize it or not, this is how all proficient
programmers think. It turns out that \textbf{the ``purpose" of almost any line
of code is to change the contents of memory in a particular way.} The name of
the game is recognizing what impact on memory each line of code has -- and
conversely, what line of code is required to make a particular change to
memory.

\section{Memory diagrams}

The focal point of this chapter will be the \textbf{memory diagram}, which
incorporates the UML object representations we discussed in
section~\ref{sec:UMLclasses}. A memory diagram depicts the contents of the
computer's memory at a \textit{snapshot in time.} At any given moment, as a
program is running, you could say ``Freeze!" and look at the memory diagram.
It would give you the exact state of the system at that moment.

\subsection{The stack and the heap}

A program's memory, it turns out, is divided into two realms with funny names:
``\textbf{the stack}" and ``\textbf{the heap}." It is vital to understand the
difference between the two, and which one is used for what. The stack contains
\textbf{statically-allocated} memory and the heap contains
\textbf{dynamically-allocated} memory. We'll unpack what all this means, but
first let me show you a full list of differences:

\vspace{.2in}
\begin{tabular}{c|c}
\textbf{stack} & \textbf{heap} \\
\hline
statically-allocated & dynamically-allocated \\
contains named things & contains unnamed things \\
contains primitive types and references & contains objects\footnote{This is
true in Java, but C++ permits programmers to store objects on the
\textit{stack} as well as the heap. I will argue that's universally dumb, and
that is a large part of what makes programming in C++ difficult: you have to
account for that happening, which requires a ton of tedious and error-prone
bookkeeping.} \\
items have a limited lifespan & items have an unlimited lifespan\footnote{Not
completely unlimited, but things on the heap stick around as long as they're
needed, rather than evaporating at the end of their current function.} \\
\end{tabular}
\vspace{.2in}

This is best understood by example, and in fact can be illustrated with just a
small function:

\begin{figure}[ht]
\begin{Verbatim}[fontsize=\small,samepage=true,frame=single]
void illustration() {
    int year = 2018;
    Car minivan = new Car("Toyota","Sienna");
}
\end{Verbatim}
\vspace{-.3in}
\caption{Some surprisingly complex code.}
\label{fig:firstCode}
\end{figure}

This teensy function, when it runs, produces memory contents as depicted in
Figure~\ref{fig:stackHeap1}. Let's go through it carefully.

\begin{figure}[ht]   % 590x220
\centering
\includegraphics[width=1\textwidth]{stackHeap1.pdf}
\caption{The stack and the heap.}
\label{fig:stackHeap1}
\end{figure}

The first line of \texttt{illustration()} creates a simple integer variable
and sets it equal to 2018. Since an \texttt{int} is a \textbf{primitive
type}\footnote{If you've never heard this lingo, a ``primitive type" is one of
the very basic lower-case Java variable types, like \texttt{int},
\texttt{double}, or \texttt{boolean}. Importantly, a primitive type is
\textit{not} an object.}, it is stored on the stack. ``On" the stack should
make you think of layering items vertically on a surface. Before this line of
code executed, nothing existed in the program's memory at all, so the stack
was nothing but a bare floor (think of it as a horizontal line). Our first
variable goes right on top of that floor.

There's a ton packed into that second line of code, so hold on to your seats.
The first thing to realize is that \textit{it encompasses both stack and
heap.} We have a named \textbf{reference variable} called \texttt{minivan},
which, as with all named things, goes on the stack (right on top of
\texttt{year}). A ``reference variable" means a variable that has the ability
to reference (or ``refer to," or ``point to") an object. However, the object
itself is created in the heap, because in Java that's where all objects live.
The word \texttt{new} is a ``heap word": using it is the only way to make an
object at all, and therefore, the only way to make something on the heap.
Finally, to carry out the equals sign (``\texttt{=}") in that line of code, we
draw an arrow from \texttt{minivan} to the object to indicate that's what it's
currently referring to.

\subsubsection{(...a brief commercial...)}

Now before we go any further, I want to \textit{\textbf{implore}} you that
\textit{this stuff does actually matter.} It's tempting at this point to think
that all this gibberish about stack and heap and whether something's on the
left or right side of a diagram is irrelevant.  Nothing could be further from
the truth. As soon as our example gets even moderately complicated, you will
absolutely get the wrong answer if you conflate or confuse the two memory
realms, or fail to keep their contents utterly in sync. Trust me on this.

\subsubsection{(...okay, back to work...)}

Okay, now a head-scratcher. Look at Figure~\ref{fig:stackHeap1} again. What
would you answer if I asked you, ``what's the \textit{name} of that blue
object?"

If you're like 99\% of novice programmers (including myself, long ago), you
would confidently answer, ``\texttt{minivan}. Its name is \texttt{minivan}."
That seems to make perfect sense. But unfortunately it is \textit{wrong}. The
truth is that \textit{the object has no name.}

Again, you may think I'm being pedantic. Let me demonstrate why I'm not.
Suppose we expanded our previous code with four more lines:

\begin{figure}[ht]
\begin{Verbatim}[fontsize=\small,samepage=true,frame=single]
void illustration() {
    ...
    Car other = new Car("Ferrari","F355");
    Car t = minivan;
    minivan = other;
    other = t;
}
\end{Verbatim}
\vspace{-.3in}
\caption{(Continuing the previous example.)}
\label{fig:additionalCode}
\end{figure}

Let's deal with the first two of these lines. The first one creates a new
reference variable called \texttt{other} on the stack, and points it to a
brand new \texttt{Car} object (unrelated to our Toyota Sienna) in the heap.
Notice that unlike with the stack, I didn't carefully put the new \texttt{Car}
exactly on top of the first one. Instead, I just threw it in there helter
skelter. This is how the heap works, and in fact why it's called a ``heap":
it's a disorganized mess of stuff that comes and goes in response to the
program's unpredictable needs. The stack is as tidy as the Library of
Congress; the heap is a teenage boy's room. Though weird, it turns out that
things have to be that way.

The second line creates a new stack variable called \texttt{t} but
emphatically does \textit{not} create a new \texttt{Car} object. Let that sink
in deeply. Many programmers, upon seeing a line begin with ``\texttt{Car t =
...} would naturally assume that line is making a new \texttt{Car}. But it's
actually only creating another variable that has the \textit{potential} to
refer to a \texttt{Car}. And in fact, after the equals sign, we do point it to
a \texttt{Car}...but one of the ones we've already instantiated (namely, the
Sienna.)

The result of executing these two lines is shown in
Figure~\ref{fig:stackHeap2}. Stare very carefully at that figure and mull over
each box and line. We have four named variables, three of which are of type
\texttt{Car}, and yet there are only \textit{two} \texttt{Car} objects because
we only executed two \texttt{new}'s. And two of our named variables --
\texttt{t} and \texttt{minivan} -- are pointing to \textit{the same object}.
This turns out to be okay. We'll have multiple references to the same object
all the time, and it's entirely healthy. What's critical not to miss is that
\texttt{t} and \texttt{minivan} are not referring to identical copies of the
\texttt{Car}, but literally \textit{the same \texttt{Car}}. If we were to
change the state of \texttt{t}'s \texttt{Car} by, say, increasing its odometer
instance variable, \texttt{minivan} would instantly experience the same
change. And that's because they \textit{are} the same. 

\begin{figure}[ht]
\centering
\includegraphics[width=1\textwidth]{stackHeap2.pdf}   % 620x230
\caption{After executing the first two lines of code
listing~\ref{fig:additionalCode}.}
\label{fig:stackHeap2}
\end{figure}


Okay, now the punchline of this whole example. I'm going to complete the
bait-and-switch, just to prove I was right before when I said ``the name of
that first blue box is \textit{not} \texttt{minivan}." Let's do the
\textit{second} two lines of code in listing~\ref{fig:additionalCode}:

\begin{Verbatim}[fontsize=\small,samepage=true]
    minivan = other;
    other = t;
\end{Verbatim}

The result of those two operations is to change what the \texttt{other} and
\texttt{minivan} variables are pointing to. Memory now looks like
Figure~\ref{fig:stackHeap3}. And so I ask you again: ``what's the
\textit{name} of that Toyota Sienna object?" I think you'll agree that
\texttt{minivan} is most certainly \textit{not} its name. Two valid ways to
refer to it are \texttt{t} and \texttt{other}, both of which point to it. But
neither one is its name. Objects simply have no name.

Names are ephemeral, momentary: they're only used temporarily so we can get at
the stuff in the heap, which is ultimately what matters.

\begin{figure}[ht]
\centering
\includegraphics[width=1\textwidth]{stackHeap3.pdf}   % 620x230
\caption{Finally, after executing the rest of code
listing~\ref{fig:additionalCode}.}
\label{fig:stackHeap3}
\end{figure}

Let me conclude this example by explaining what I meant earlier about
``limited lifespans." After executing the ``\texttt{other = t;}" line, we are
done with the function. It's time to return control to whoever called
\texttt{illustration()} in the first place. And at this point, all of our
named variables -- \texttt{t}, \texttt{other}, \texttt{minivan}, and even
\texttt{year} -- cease to exist. Their destiny was only to provide service
during the time that \texttt{illustration()} was being executed. 

But the stuff on the heap lives on after. Long after a function is completed,
the objects it may have created or changed have a presence that will affect
the behavior of other, future functions. In this case, since we weren't passed
any arguments and didn't return anything, our Toyota and Ferrari \textit{will}
actually peacefully go away. But in general there are meaningful, long-term
effects, and in the next section we'll see an example in action.

Most methods are just like this. They create a few named variables so they can
change the contents of the heap in some way, and then clean up their dishes
and return with the heap thus changed. That is their \textit{raison d'etre}.
It's a short but happy life.





\chapter{Blueprints: UML class diagrams}
\label{ch:blueprints}

We spent last chapter discussing the \textbf{dynamic} view of a program: what
happens to memory, step by step, as it unfolds. In this chapter, we'll switch
to a \textbf{static} view: long-term, what are the program's classes, methods,
and relationships between them?\footnote{The words ``dynamic" and ``static"
are ubiquitous in computer science, and mean a zillion different unrelated
things. For example, we've already seen the Java ``\texttt{static}" keyword,
and how it indicates class-level rather than an object-level ownership. We've
also hinted at the stack having ``statically-allocated memory" and the heap
being ``dynamically-allocated." These terms are \textbf{\textit{unrelated}} to
our use of the words in this chapter. At present, by ``dynamic" we mean ``the
contents of memory changing as the program runs"; and by ``static" we mean
``the consistent, permanent characteristics of a program, quite apart from how
it might be behaving at any moment, which include its classes, methods, and
associations."}

If there's a type of UML diagram that deserves the name ``blueprint," it's the
\textbf{class diagram}. Class diagrams depict a high-level, stable perspective
of a software system. When you want to figure out how a large OO program
works, or when you're tasked with implementing a system that someone else has
designed, the first thing you look at are its class diagram(s).

UML class diagrams contain a number of elements, each of which has a very
specific meaning. We'll cover each in turn.

\section{Classes}

Unlike memory diagrams, which depict objects, class diagrams contain classes
(duh). We've already seen what a single class looks like in
section~\ref{sec:UMLclasses} (\textit{e.g.}, the left side of
Figure~\ref{fig:classObject}.) Most class diagrams contain many such classes.
Recall that each class has three compartments, containing the class's name,
its inst vars, and its methods, in that order.

By the way, one flexible (yet slightly annoying IMO) aspect of UML is that it
allows \textbf{varying levels of detail}. In other words, on a particular
diagram, you may or may not want to show all the instance variables and
methods, because it may or may not be relevant to the purpose of that
particular diagram. Similarly, you may or may not want to show all the aspects
of each inst var or method; perhaps it's too early in the design process to
completely specify all the parameters and return types, for example. To
illustrate, all three pictures in Figure~\ref{fig:graceful} are legit
ways of representing the \texttt{Car} class. We can include as much or as
little detail as we please.

\begin{figure}[ht]
\centering
\includegraphics[width=1\textwidth]{graceful.pdf}   % 670x320
\caption{Three equally valid ways to draw the \texttt{Car} class on a class
diagram, depending on how much detail it makes sense to include.}
\label{fig:graceful}
\end{figure}

The reason I find this annoying, by the way, is that it's ambiguous. If you
see no inst vars in the second box, does that mean (a) that class \textit{has}
no inst vars, or (b) the designer didn't think it was relevant to include them
on this particular diagram? No way to really know.

\section{Associations}

Perhaps the most important bits of information on a class diagram are the
\textbf{association}s between classes. An association means that two classes
collaborate together in some way to achieve some larger purpose. It is
indicated on a class diagram by a line connecting the two classes. Different
types of lines represent different kinds of relationships between the classes.
It's important not to mix them up, because if you do, you're dictating
something incorrect to the programming team about how the classes are intended
to work.

\begin{figure}[ht]
\centering
\includegraphics[width=0.7\textwidth]{assocArrows.pdf}   % 350x200
\vspace{.2in}
\caption{Diagrammatic elements for different association types.}
\label{fig:assocArrows}
\end{figure}

\subsection{Dependency associations}

Figure~\ref{fig:assocArrows} shows some of the UML association arrows and
their meaning. (There are others we'll get to in future chapters.) The dashed
line with a crow's foot arrowhead is called a \textbf{dependency}, and is the
``weakest" of the association types. When I say weak, I mean that the
relationship between the two classes isn't as important, nor as permanent, as
with the other association types we'll discuss later.

A dependency between classes \texttt{A} and \texttt{B} can be thought of in a
couple of ways:

\begin{compactitem}
\item One or more methods of the \texttt{A} class will \textit{call methods
on} a \texttt{B} object.
\item The \texttt{A} class \textit{is dependent on the interface of} the
\texttt{B} class.
\end{compactitem}

The word \textbf{interface} -- like stack, heap, dynamic, static, and many
other computer science words -- has multiple meanings. We'll talk about the
Java \texttt{interface} keyword later in the book. For now, when I say
interface I mean \textit{those aspects of a class that a user of that type of
object can see.} This boils down to: the methods you can call on it, together
with their argument lists and return types. Specifically, the interface does
\textit{not} include the method implementations (the bodies of the methods),
nor the instance variables.

If you think about it, you'll realize why the above two bullet points are
actually equivalent. Suppose some class \texttt{A} method has this line of
code in it: ``\texttt{String s = B.scissorKick(15)}". Then clearly the code in
the \texttt{A.java} file is \textit{dependent} on the fact that class
\texttt{B} has a \texttt{.scissorKick()} method, and that it takes an integer,
and returns a \texttt{String}. If any of that ever changed in the
\texttt{B.java} file, then class \texttt{A} would be impacted.

\begin{figure}[ht]
\centering
\includegraphics[width=0.8\textwidth]{dependencyAssoc.pdf}   % 
\caption{Examples of dependency associations.}
\label{fig:dependencyExamples}
\end{figure}

The strange-looking words adjacent to the dependency arrows in
Figure~\ref{fig:assocArrows} go by the even stranger-sounding term
\textbf{stereotypes}. A stereotype in UML is an extra bit of information that
enhances part of a diagram (an association arrow, as here, or sometimes a
class, method, or other element) by making its meaning more precise.
Stereotypes are usually displayed enclosed by double-wakkas
(``$\ll$...$\gg$").

In the case of dependency associations, the stereotype ``$\ll$uses$\gg$" means
pretty much what a dependency always means: that the designer intends class
\texttt{A} to ``use" (\textit{i.e.}, get its hands on, and call method(s) on)
object(s) of class \texttt{B}. The ``$\ll$instantiates$\gg$" stereotype goes a
bit further, and implies that some method of \texttt{A} will
\textit{instantiate} \texttt{B} objects in addition to merely calling methods
on them.

The examples in Figure~\ref{fig:dependencyExamples} are from a Dungeons \&
Dragons type combat simulator. A \texttt{Battle} object represents a fight
between adventurers and monsters. While simulating this fight, a
\texttt{Battle} will make use of one or more \texttt{Die} (singular of ``dice")
objects to roll random numbers that determine the outcome. This is a
``$\ll$uses$\gg$" association, since \texttt{Battle}'s code now depends on
\texttt{Die}'s interface not changing.

Elsewhere in the program, wizards sometimes cast ranged spells, like fireballs
or lightning bolts, to damage distant enemies. In the simulator, a
\texttt{Wizard} object might therefore instantiate a \texttt{RangedSpell}
object to carry out this attack. Since somewhere in the \texttt{Wizard} class's
code there will be a ``\texttt{new RangedSpell()}" line, we say that
\texttt{Wizard} $\ll$instantiates$\gg$ \texttt{RangedSpell}.

\subsubsection{Dependencies in code}

Now what would we expect to see in the code that would reflect this kind of
association? In the ``$\ll$uses$\gg$" case, we expect to see one or more
methods of the \texttt{A} class making method calls on \texttt{B} objects.
Perhaps something like this:

\begin{Verbatim}[fontsize=\scriptsize,samepage=true,frame=single]
class Battle {
    ...
    void resolveAttack(Adventurer a, Monster m, Die d) {
        ...
        if (d.roll() < a.currentWeapon().attackStat()) {
            ...
        }
    }
    ...
}
\end{Verbatim}

The design diagram doesn't specify exactly what \texttt{A} method will be
called where, just that method calls are expected. This communicates something
important to the programmer.

For ``$\ll$instantiates$\gg$", we'd expect to see the word \texttt{new}
somewhere in \texttt{A}:

\begin{Verbatim}[fontsize=\scriptsize,samepage=true,frame=single]
class Wizard {
    ...
    void takeAction(ArrayList<Monster> enemies) {
        ...
        if (enemies.size() > 3) {
            RangedSpell fireball = new RangedSpell("Fireball", 60, 12);
            fireball.cast();
            ...
        }
    }
    ...
}
\end{Verbatim}

\subsection{``Has-a" associations}

The next strongest type of association has a bizarre name: it's called
``\textbf{has-a}." We denote it with a solid arrow between classes, with a
crow's foot on one side or both.

When class \texttt{A} has-a class \texttt{B}, that is nearly always a signal
to the programmer that \texttt{A} should have an \textit{instance variable} of
type \texttt{B}.\footnote{Or perhaps a \textbf{collection} of \texttt{B}
objects rather than a single \texttt{B} object, as we'll see later in the
chapter.} In other words, not only does an \texttt{A} object call
methods on a \texttt{B} (as in the dependency association), but an \texttt{A}
object actually holds on to one (or more) \texttt{B} objects for the
long-haul.

Now in some cases, the ``has-a" verbiage makes perfect sense. If our Domino's
Pizza delivery manager application had a \texttt{Pizza} class and a
\texttt{Topping} class, it would be no-brainer to say that every
\texttt{Pizza} has-a \texttt{Topping}. It conjures up in our minds a picture
of containment, or ownership. Perfect. However, we also use this type of
association in cases where containment doesn't make sense at all.

For example, in the same application it would be quite sensible to say that
``every \texttt{Pizza} has-a \texttt{DeliveryCar}." But obviously the delivery
car isn't ``inside" the pizza in the same physical way that the toppings are
inside it. So what does it mean then?

\begin{figure}[ht]
\centering
\includegraphics[width=1\textwidth]{wrongRightHasA.pdf}   % 
\caption{The \textbf{wrong}, and \textbf{right}, way of visualizing a ``has-a"
association in Java.}
\label{fig:wrongRightHasA}
\end{figure}

The key is making sure you have the right mental model.
Figure~\ref{fig:wrongRightHasA} shows both the wrong, and the right, way to
envision a has-a relationship (at least, in Java). In memory, there is
\textit{no} ``containment" as in the left-hand (wrong) image. The
\texttt{Topping} object isn't enclosed inside the \texttt{Pizza}, or even
exclusively owned by it. It's simply pointed to by one of the \texttt{Pizza}
object's inst vars. The right-hand side of the figure is the correct one --
and I daresay it's not problematic at all to think of a \texttt{Pizza}
``having" a \texttt{DeliveryCar} in this way. All it really means is that a
\texttt{Pizza} object ``knows about" a \texttt{DeliveryCar}, which is the
particular car that's delivering it.

Another reason that the correct mental model of ``has-a" is important is that
it is possible, and even common, for the association to go \textit{both ways}.
We use the term \textbf{navigability} for the question ``which direction does
the arrow go -- from \texttt{A} to \texttt{B}, from \texttt{B} to \texttt{A},
or both?" When it goes both ways, we call it a \textbf{bidirectional}
association.

\begin{figure}[ht]
\centering
\includegraphics[width=1\textwidth]{bidirectional.pdf}   % 750x235
\caption{A bidirectional ``has-a," depicted on a class diagram (left) and a
memory diagram.}
\label{fig:bidirectional}
\end{figure}

An example is the left-hand side of Figure~\ref{fig:bidirectional}. Here, our
\texttt{Driver} class and our \texttt{DeliveryCar} class each know about the
other, and in fact each hold on to an instance variable of the other type. If
we viewed this \texttt{A}-having-an-instance-variable-of-type-\texttt{B} thing
as the \texttt{A} object \textit{enclosing} the \texttt{B}, we'd blow a fuse.
\texttt{A} would contain \texttt{B}, which would contain \texttt{A}, which
would contain \texttt{B}, which... That way madness lies. But notice that
nothing paradoxical happens at all in the corresponding memory diagram on the
right-hand side of the figure. Each object points to the other, so that a
\texttt{Driver} object knows which \texttt{DeliveryCar} he/she is driving, and
a \texttt{DeliveryCar} also knows which \texttt{Driver} is driving it. No
biggie.

\begin{figure}[ht]
\centering
\includegraphics[width=1\textwidth]{wrongHasA.pdf}   % 640x180
\caption{One wrong way to model an instance variable. The ``has-a" arrow
already indicates that every \texttt{Pizza} has-a \texttt{Topping}: the
extraneous \texttt{topping} entry in the \texttt{Pizza} class's second box is
redundant and incorrect.}
\label{fig:wrongHasA}
\end{figure}

Note, by the way, that the has-a arrow implies the existence of the inst var
\textit{all by itself}. The class diagram should \textit{not} contain a
duplicate copy of the inst var in its second compartment. That would be
redundant, and is considered an error (see Figure~\ref{fig:wrongHasA}).

\pagebreak
\subsubsection{``Has-a" associations in code}

Obviously instance variables are how ``has-a" associations are manifested in a
Java program. For \texttt{Pizza} and \texttt{Topping}, we'd see:

\begin{Verbatim}[fontsize=\scriptsize,samepage=true,frame=single]
class Pizza {
    ...
    Topping topping;
    ...
}
\end{Verbatim}

and for our bidirectional \texttt{Driver}/\texttt{DeliveryCar}, we'd see both
\begin{Verbatim}[fontsize=\scriptsize,samepage=true,frame=single]
class Driver {
    ...
    DeliveryCar car;
    ...
}
\end{Verbatim}

and

\begin{Verbatim}[fontsize=\scriptsize,samepage=true,frame=single]
class DeliveryCar {
    ...
    Driver currentDriver;
    ...
}
\end{Verbatim}

These examples both assume that a \texttt{Pizza} has only \textit{one}
\texttt{Topping}, \textit{etc.} If this isn't so, we'd use some kind of
container class instead:

\begin{Verbatim}[fontsize=\scriptsize,samepage=true,frame=single]
class Pizza {
    ...
    ArrayList toppings;
    ...
}
\end{Verbatim}

More on that later.

\subsection{Aggregation associations}

Continuing on towards the ``stronger" end of the association continuum, an
\textbf{aggregation} implies exclusive ownership of the object(s) in question.
In other words, if \texttt{A} aggregates \texttt{B}, not only does it mean
that \texttt{A} has an instance variable of type \texttt{B}, but that
\textit{no other \texttt{A} object also has that \texttt{B}.}

This is frequently misinterpreted, so let me expand on that. The
``exclusivity" thing is a statement about \textit{objects}, not classes. If
\texttt{A} aggregates \texttt{B}, that does \textit{not} mean that no other
class can have an instance variable of type \texttt{B}. Rather, it means that
if a particular \texttt{B} object is pointed to by an \texttt{A} object, no
\textit{other} \texttt{A} object also points to that \texttt{B}.

\begin{figure}[ht]
\centering
\includegraphics[width=1\textwidth]{aggregationAssoc.pdf}   % 620x220
\caption{Examples of aggregation associations.}
\label{fig:aggregationAssoc}
\end{figure}

Examples appear in Figure~\ref{fig:aggregationAssoc}. Note that the diamond
appears on the ``aggregator" side; \textit{i.e.}, adjacent to the class that
will have the instance variable.

In the first example, for a Banner-like college enrollment management system,
each \texttt{Professor} will teach some number of \texttt{Section}s in a given
semester. If Professor Jones is assigned to teach section 03 of BIOL 121, then
no \textit{other} professor is also assigned to that section. That's what the
white diamond communicates.

In the second example, from a Facebook-like social networking site, users can
arrange their \texttt{Photo}s into \texttt{Album}s. As indicated on this
diagram, a given \texttt{Photo} is \textit{not} intended to simultaneously
belong to more than one \texttt{Album}. (If we wanted to relax that
constraint, and permit photos to belong to multiple albums at once, we would
get rid of our white diamond and use a plain-old ``has-a" arrow instead.)

\subsubsection{Aggregations in code}

Aggregation is intended to imply some sort of collection or ownership
relationship between the two classes. However, in terms of the Java code that
you initially write, \textit{there is no immediate difference between an
aggregation and a regular ``has-a."} In both cases, you'll make an inst var of
the appropriate type in the appropriate place. The code difference between
aggregation and has-a won't come out until later, when the class methods are
being implemented. That white diamond is more of a long-term signal to the
programmer about how two classes are generally intended to operate together,
rather than being a cue to write the first bit of code differently than you
otherwise would.

\subsection{Composition associations}

The last association type we'll cover, and the most tightly-binding between
classes, is called \textbf{composition}. It's a lot like aggregation (even the
diamond syntax is the same, except it's black) but with one difference. With
composition, not only does an \texttt{A} object have exclusive ownership over
its \texttt{B} object(s), but there's a \textbf{lifespan dependency} as well:
if the \texttt{A} ever disappears, its constituent \texttt{B}'s should also
cease to exist.

\begin{figure}[ht]
\centering
\includegraphics[width=1\textwidth]{compositionAssoc.pdf}   % 700x230
\caption{Examples of composition associations.}
\label{fig:compositionAssoc}
\end{figure}


Consider the examples in Figure~\ref{fig:compositionAssoc}. In this social
networking site, every \texttt{User} has a \texttt{Profile}. That
\texttt{User} is the \textit{only} one with that particular profile (hence
this is at least aggregation) and what's more, \textit{the \texttt{Profile}
has no meaningful existence without its \texttt{User}.} If the user ever
deletes their account, it wouldn't make sense to have a disembodied
\texttt{Profile} object lying around, so it should automatically disappear as
well. This lifespan connection is really the only difference between the white
diamond and the black.

On the right-hand side is an example from some kind of email reader
application (like Outlook, gmail, or Thunderbird). A user can compose an
\texttt{Email} with some text and a list of recipients, and then add
\texttt{Attachment}s to it to send images, documents, code, \textit{etc.} But
what if the user decides to abandon the message before sending it? The
\texttt{Email} object should go away, but its \texttt{Attachment}s should too.
Hence this is another example of composition.

\subsubsection{Compositions in code}

Just as with aggregations, there's no simple Java keyword that magically maps
to the idea of ``composition." Instead, the presence of the black diamond
suggests to the programmer the intended function of the classes involved, and
she will write the code with this in mind.


\subsection{Association annotations}

As if all this weren't enough, there are also a couple more syntactic things
to learn about UML associations. An \textbf{annotation} is another mark on
part of a diagram that gives more detail about how it is to be understood or
implemented. We've already seen two examples of this: the stereotypes we
included next to dependency lines are a type of annotation, as are the
arrowheads to indicate navigability. We'll learn two more in this section.

\subsubsection{Multiplicity}

The \textbf{multiplicity} of an association indicates \textit{how many}
objects are involved in each concrete relationship. It's important to realize
that even though multiplicity is shown on a class diagram, it's really a
statement about \textit{objects}.

Let's start with the left-most example in Figure~\ref{fig:multiplicity}. There
we have two classes from a DMV software system, connected with a ``has-a"
association between \texttt{Driver} and \texttt{License}, navigable both ways.
Note the numeral ``\textbf{1}" annotation both sides of the arrow. This
indicates that \textit{every \texttt{Driver} ``goes with" just one
\texttt{License} object}, and \textit{every \texttt{License} also goes with
just one \texttt{Driver}.} This is called a \textbf{one-to-one association},
sensibly enough.

\begin{figure}[ht]
\centering
\includegraphics[width=1\textwidth]{multiplicity.pdf}   % 790x270
\caption{Association annotations indicating multiplicity.}
\label{fig:multiplicity}
\end{figure}

In the center example, on the other hand, we have a ``$\star$" on the side of
the arrow that connects to \texttt{Weapon}. In UML, the symbol ``$\star$"
means \textbf{zero or more}. So here's how we interpret this
\textbf{one-to-many association}: every \texttt{Adventurer} has zero or more
weapons, while every \texttt{Weapon} is possessed by just one
\texttt{Adventurer}. Note that since the direction is only navigable in one
direction, this indicates that although an \texttt{Adventurer} is aware of
which \texttt{Weapon}s she owns, the \texttt{Weapon} objects are \textit{not}
aware of which \texttt{Adventurer} owns them. This knowledge (or lack thereof)
is perfectly okay, and does not invalidate the meaning of the \textbf{1} or
the $\star$ in the slightest.

Finally, on the right side, we have a \textbf{many-to-many association}
between \texttt{Tran\-script} and \texttt{Course}. This says that every
\texttt{Transcript} object is associated with potentially multiple
\texttt{Course} objects, while each \texttt{Course} object appears on more
than one \texttt{Transcript}. Navigability-wise, in this example
\texttt{Transcript}s maintain a record of which \texttt{Course}s they contain,
whereas \texttt{Course} objects don't know which \texttt{Transcript}s they
appear on (if any).

You'll occasionally see more elaborate multiplicity notations on class
diagrams. The notation ``\textbf{0..$\star$}" means ``zero or more"...which is
of course exactly what plain old ``$\star$" means. The only reason for a
designer to write ``\textbf{0..$\star$}" is for emphasis: she is stressing to
the coding team that an object of the first type may well have \textit{zero}
objects of the second type at any given time; this is a real possibility. In
contrast, if she writes ``\textbf{1..$\star$}" that means ``one or more,"
which signals the coder ``by the way, every object of the first type should
\textit{always} be assigned to at least one object of the second type; you
should keep that in mind as you code." Even more rarely, you'll see
multiplicities like ``\textbf{5}" (``each object of type A is associated with
exactly five objects of type B"), or ``\textbf{3..8}" (``each object of type A
is associated with anywhere from three to eight objects of type B"),
\textit{etc.} These are rare, especially since as we'll see in the next
section, there really isn't any way to code those constraints explicitly in a
language like Java.



\subsubsection{Multiplicity in code}

So what does all this look like in code? Well, first remember that inst vars
are only used in the direction(s) along which the association is navigable.
For Figure~\ref{fig:multiplicity}, this means that only \texttt{Driver},
\texttt{License}, \texttt{Adventurer}, and \texttt{Transcript} will have inst
vars related to these associations; \texttt{Weapon} and \texttt{Course} will
not. Furthermore, if the multiplicity is a \textbf{1}, the inst var will be of
the type the arrow is pointing to; if it's a $\star$, it will be \textit{some
collection} of that type. Which sort of collection is used -- an array, an
\texttt{ArrayList}, a \texttt{Hashtable}, a \texttt{Set}, \textit{etc.}~-- is
normally up to the programmer, and is decided based on the run-time
performance features of that collection type.

So here's some code we might reasonably expect to see from our three examples:

\begin{Verbatim}[fontsize=\small,samepage=true,frame=single]
    class Driver {                        class Adventurer {
        String name;                          String name;
        License license;                      int hitPoints;
        ...                                   ArrayList<Weapon> weapons;
    }                                         ...
                                          }
    class License {
        String number;                    class Transcript {
        Driver owner;                         Course[] courses;
        ...                                   ...
    }                                     }
\end{Verbatim}

Here the programmer of the \texttt{Adventurer} class has chosen to use an
\texttt{ArrayList} to hold each adventurer's weapons, while the
\texttt{Transcript} author decided on a simple array. In terms of being
faithful to the design, neither choice is right or wrong.

\subsubsection{Roles}

Our last type of association annotation has to do with \textbf{roles}.
Sometimes, a design will be specific not only about the \textit{existence} of
the association between two classes, and about which-knows-about-which, and
about how-many-are-involved, but also the intended \textit{meaning} of the
relationship. In other words, it may specify what role each of the object
types is expected to play with respect to the other. This may sound a bit
abstract, but some examples will make it clearer.

\begin{figure}[ht]
\centering
\includegraphics[width=1\textwidth]{roles.pdf}   % 670x400
\caption{Association annotations indicating roles.}
\label{fig:roles}
\end{figure}

The upper-left example in Figure~\ref{fig:roles} shows a piece of a Marvel
comic book database application. We have \texttt{Hero} and \texttt{Villain}
classes, and a one-to-one association between them...but what does the
association \textit{mean}? If \texttt{Hero X} ``goes with" \texttt{Villain Y},
does that mean that \texttt{X} has recently beaten up \texttt{Y}? That
\texttt{X} admires \texttt{Y}? That \texttt{X} secretly \textit{is}
\texttt{Y}, unbeknownst to the public?

The word ``archnemesis" next to the \texttt{Villain}-side of the arrow spells
it out. It's called a \textbf{role name}. It tells us that in this
relationship, the role that the \texttt{Villain} plays with respect to the
\texttt{Hero} is that the former is the archenemy of the latter.

Moving to the right side of the diagram, we have an interesting situation
involving only one class: \texttt{TwitterUser}. This class apparently has an
association to itself! This turns out not to be as weird as it might seem. In
fact, if you think about a social network like Twitter, the most meaningful
relationships \textit{are} between objects of the same class. And that's the
key to de-weirding it in your mind: remember that an association is a
statement about \textit{objects}, not classes. We're not saying
``\texttt{TwitterUser} has a relationship with itself" but rather ``each
\texttt{TwitterUser} is related to zero or more other \texttt{TwitterUser}s."

And what do those relations mean, you ask? The role name tells us: one of the
users ``follows" the other in the Twitter sense. In this diagram, we have role
names on both sides of the arrow, although that's probably not strictly
necessary. What is interesting here is the navigability of the association:
according to the design, a \texttt{TwitterUser} object is aware of what other
\texttt{TwitterUser}s follow him/her, but not which \texttt{TwitterUser}s
he/she follows. If the design team decided they needed to track that
separately, they'd need another arrowhead on the top side of the line.

Finally, the bottom example illustrates two \textit{different} associations
between the same two classes. This can happen as well. In this case, there are
two distinct roles that \texttt{Professor}s play with respect to
\texttt{Student}s: as their instructors (each student has several) and as
their advisor (each student has one). The role names are imperative here,
since otherwise the programming team would be lost as to why there are two
relationships and what each one is supposed to mean.

\subsubsection{Roles in code}

Often, the role name on the diagram is simply used as the instance variable
name in the code. For instance, I'd expect to see something like this:

\begin{Verbatim}[fontsize=\small,samepage=true,frame=single]
class Student {
    String major;    
    Professor advisor;
    ArrayList<Professor> instructors;
}
\end{Verbatim}

since those names were handed to us on a silver platter in the design diagram.

\section{Visibility}

The other parts of UML class diagrams that we'll annotate with extra
information will indicate the level of \textbf{visibility} that the designer
intends the various inst vars, methods, and even classes to possess.
Visibility has to do with promoting \textbf{encapsulation}, perhaps the most
important of all OO principles, which will be the subject of an entire future
chapter. For now, let me just give you the syntax and the operational
implications of the different visibility levels.

This is one area where our UML diagrams, ostensibly
programming-language-neutral, will betray a very Java-ish flavor. That's
because in Java, there are four specific visibility levels for methods and
inst vars (two for classes), each with a precise meaning, and we'll have UML
syntax to indicate each. The complete list can be found in the tables in
Figures~\ref{fig:visibilityLevels} and \ref{fig:visibilityLevelsClasses}. In
both tables, the visibility levels are listed in order from most restrictive
to least restrictive.

\begin{figure}[ht]
\centering
\begin{tabular}{c|c|c|l}
\thead{visibility level} & \thead{Java keyword} & \thead{UML syntax} &
\thead{visible to...} \\
\hline
private & \texttt{private} & \textbf{--} & the class itself \\
package & (none) & \textbf{\freakingtilde} & any class from the same package\\
protected & \texttt{protected} & \textbf{\#} & \makecell{from the same
package, or a subclass\footnote{A ``subclass" has to
do with the topic of \textbf{inheritance} in OO, a subject of a future
chapter. For now, I just want to make the table complete.}} \\
public & \texttt{public} & \textbf{$\plus$} & any method anywhere \\
\end{tabular}
\vspace{.1in}
\caption{The four Java visibility levels for \textit{methods and inst vars}.}
\label{fig:visibilityLevels}
\end{figure}

\begin{figure}[hb]
\centering
\begin{tabular}{c|c|c|l}
\thead{visibility level} & \thead{Java keyword} & \thead{UML syntax} &
\thead{visible to...} \\
\hline
non-public & (none) & (none) & only classes in the same package \\
public & \texttt{public} & \textbf{$\plus$} & any class anywhere \\
\end{tabular}
\vspace{.1in}
\caption{The two Java visibility levels for \textit{classes}.}
\label{fig:visibilityLevelsClasses}
\end{figure}

Now it's important to understand that unlike multiplicity, visibility
modifiers make a statement about \textit{classes}, not \textit{objects}. Also,
crucially, visibility is about the very \textit{existence} of the method, inst
var, or class, not its \textit{value.} This is very commonly misconstrued, so
let me clarify with an example.

Suppose a class diagram included the class in
Figure~\ref{fig:visAboutClasses}. Here, for the first time, we see visibility
modifiers in action. In particular, the \texttt{numHits} and
\texttt{numAtBats} inst vars are both marked as private, while the
\texttt{isBetterThan()} method is public.

\begin{figure}[hb]
\centering
\includegraphics[width=0.5\textwidth]{visAboutClasses.pdf}
\caption{A class whose components bear visibility annotations.}
\label{fig:visAboutClasses}
\end{figure}

Here's the kind of code we want to make possible with this method:

\begin{Verbatim}[fontsize=\small,samepage=true,frame=single]
    ...
    Ballplayer jeter = new Ballplayer("Jeter");
    Ballplayer arod = new Ballplayer("Rodriguez");
    if (jeter.isBetterThan(arod)) {
        System.out.println("Sign Jeter to a zillion dollars!");
    }
    ...
\end{Verbatim}

Let's inspect the \textit{inside} of the \texttt{.isBetterThan()} method
(\textit{i.e.}, its implementation). Suppose it reads like
this\footnote{Apologies to baseball fans for the gross simplification of
reducing an entire player's ``goodness" down to his or her batting average. Of
course in real life there are all kinds of other stats that come into play
here -- slugging percentage, base running stats, defensive ability,
\textit{etc.} -- as well as impossible-to-quantify aspects like teamwork,
inspiration, and clubhouse chemistry.}:

\begin{Verbatim}[fontsize=\footnotesize,samepage=true,frame=single]
class Ballplayer {
    private int numHits;
    private int numAtBats;
    ...
    public boolean isBetterThan(Ballplayer other) {
        double myBattingAvg = ((double) numHits) / numAtBats;
        double otherBattingAvg = ((double) other.numHits) / other.numAtBats;
        if (myBattingAvg > otherBattingAvg) {
            return true;
        } else {
            return false;
        }
    }
    ...
}
\end{Verbatim}

Now the key line I want to draw your attention to is the \textit{second} line
of the method. It reads:

\begin{verbatim}
double otherBattingAvg = ((double) other.numHits) / other.numAtBats;
\end{verbatim}

My question to you, the dear reader, is this: do you think this line ought to
compile without errors, or no? Take a moment to consider your answer.

Many, many students assume this line will \textit{not} compile cleanly. Here's
their reasoning: ``We're calling \texttt{.isBetterThan()} on a particular
\texttt{Ballplayer} object (say, \texttt{jeter}). And we're passing another
\texttt{Ballplayer} object as a parameter (say, \texttt{arod}). Now both
\texttt{numHits} and \texttt{numAtBats} are marked \textbf{\texttt{private}}
in the class. Therefore, \texttt{arod}'s values for these should be protected
from, and unavailable to, the \texttt{jeter} object. It stands to reason that
this will not be allowed. Otherwise, we'd be allowing one object to access
another object's private data."

This sounds so eminently reasonable, and yet it is dead wrong. Here's why:

\begin{itemize}
\itemsep.1em
\item A ``private" inst var does \underline{not} mean that one object's
\textit{value} is hidden from another \textit{object}.
\item A ``private" inst var \underline{does} mean that the very
\textit{existence} of one class's inst var is hidden from other
\textit{classes}.
\end{itemize}

In other words, it's not a ``data privacy" thing like keeping your information
inaccessible to creepy people online. Instead, it's a \textit{code
encapsulation thing} that prevents one class from making (and thereafter
depending upon) assumptions about another class's design decisions. In terms
of the online creep example, here's how I'd explain it:

\begin{itemize}
\itemsep.1em
\item Making \texttt{SSN} (Social Security Number) a private inst var of the
\texttt{Person} class does \underline{not} mean that one person cannot find
out another person's SSN.
\item Making \texttt{SSN} a private inst var of the \texttt{Person} class
\underline{does} mean that \texttt{Dog}s, \texttt{Website}s,
\texttt{CreditCard}s, \textit{etc.}~don't even know that people \textit{have}
Social Security Numbers at all.
\end{itemize}

In yet other words, visibility is about \textit{variables} and
\textit{classes}, not \textit{values} and \textit{objects}.

We'll see why this is useful in the encapsulation chapter. For now, just
realize that the code above \textit{does} compile cleanly for one simple
reason: it's a method of the \texttt{Ballplayer} class. Any method of the
\texttt{Ballplayer} class can talk about any inst var or method of the
\texttt{Ballplayer} class, regardless of which particular object is in view.

To complete the example, here's some code which indeed does \textit{not}
compile because of those private \texttt{numHits} and \texttt{numAtBats}:

\begin{Verbatim}[fontsize=\footnotesize,samepage=true,frame=single]
class Team {
    private ArrayList<Ballplayer> roster;
    ...    
    public void printRoster() {
        System.out.println("Name          Hits    ABs");
        for (Ballplayer b : roster) {
            System.out.println(b.name + "     " + b.numHits + "  " + b.numAtBats);
        }
    }
    ...
}
\end{Verbatim}

\begin{samepage}
When we try to compile it, the \texttt{println()} statement inside the
\texttt{for} loop barfs with:
\footnotesize
\begin{verbatim}
Team.java:25: error: numHits has private access in Ballplayer
    System.out.println(b.name + "     " + b.numHits + "  " + b.numAtBats);
                                           ^
Team.java:25: error: numABs has private access in Ballplayer
    System.out.println(b.name + b.numHits + b.numABs);
                                             ^
\end{verbatim}
\normalsize
\vspace{-.2in}
as we would expect. It's because the offending code is a \texttt{Team} method,
not a \texttt{Ballplayer} method, and therefore cannot refer to any of
\texttt{Ballplayer}'s private components (inst vars or methods).
\end{samepage}

The same mechanic is at play with methods as it is with inst vars: no code in
a class can \textit{call} a method unless it has visibility to that method,
as specified in the rightmost column of Figure~\ref{fig:visibilityLevels}.

If you're wondering why it would ever make sense to have a private method, the
answer is: as a helper method, for other (perhaps public) methods of that
class to call internally. Having lots of short methods to perform basic tasks,
but not exposing those methods outside the class, is one sign of a good
designer.

\subsection{Which visibility level to choose}

Both inst vars and methods can technically have any of the four visibility
levels assigned to them from the table in Figure~\ref{fig:visibilityLevels}.
Here are the rules (and strong suggestions) to keep in mind:

\begin{enumerate}
\itemsep.1em
\item Always make all instance variables \texttt{private}. That's the easiest
design decision you'll ever make. As we'll see in the encapsulation chapter,
giving an inst var any other level of visibility unacceptably sacrifices
encapsulation.
\item Always make methods ``as private as possible." This promotes
encapsulation and reduces dependencies. When in doubt, err on the side of the
\textit{higher} entry in Figure~\ref{fig:visibilityLevels}, not the lower. If
it turns out you must make it more accessible later on, you can always move
its visibility lower on the chart without breaking anything. The reverse is
not true.
\end{enumerate}

Lastly, a word about package-level visibility. Recall that a Java
\textbf{package} is an organizational mechanism that allows a developer to
group related classes together. One reason to do this is to promote neatness
and tidiness throughout your code base: instead of having all your classes in
one huge directory, you divvy them up according to functional area. 

Another reason is to take advantage of package-level visibility. There may be
design decisions you make (namely, certain methods you create on a class) that
you don't necessarily want to make publicly accessible to all users of the
class, yet which it does make sense to make available to the other classes
that are collaborating with that class. Package-level visibility was designed
for this purpose. Note that there is no Java keyword for it: it's the default.
This is because Gosling \& Co.~(the designers of Java) were proud of the
package concept and wanted to promote its use among Java developers as much as
possible. So you have to explicitly type if you want any other choice. I think
package-level visibility is a neat feature, but is underutilized.

\subsection{Class visibility}

As shown in Figure~\ref{fig:visibilityLevelsClasses}, the notion of visibility
also extends to entire classes in Java. But it's simpler: either a class is
public, or it's not. If it's public, any class anywhere can refer to it, and
if it's ``non-public" (yep, that's actually the term) it effectively has
package-level visibility (\textit{i.e.}, only other classes in its package can
use it.)

Non-public classes thus play the same sort of role as private helper methods
do: the public classes use them to help get their job done, but the non-public
ones aren't designed to be directly instantiated (or even seen) by the outside
world. Their use in practice is somewhat rarer than private methods, but I
encourage their use.


\section{Putting it all together}

All right, let's close this chapter with a small but still full-blown class
diagram that illustrates most of the above features. See if you can interpret
all of Figure~\ref{fig:fullClassDiag} correctly.

\begin{figure}[ht]
\centering
\includegraphics[width=1\textwidth]{fullClassDiag.pdf}   % 790x440
\caption{A full-blown class diagram. (The color is not part of UML; I only
colored certain elements so I could refer to them in the text.)}
\label{fig:fullClassDiag}
\end{figure}

Here's an incomplete list of things we know from the diagram. Each item's
color corresponds to an item in Figure~\ref{fig:fullClassDiag}:

\begin{enumerate}
\itemsep.1em

\definecolor{darkgreen}{rgb}{0,.65,0}
\definecolor{darkblue}{rgb}{0,0,.95}
\item \textcolor{darkgreen}{The \texttt{Ballplayer} class is public, and thus
can be used by any class in any other package. The \texttt{Simulator} class
isn't, though, and can only be referenced by classes in the same package.}

\item \textcolor{BurntOrange}{Although anyone can use a \texttt{Team} object,
only classes in this package can instantiate one. (And to do so, you must
specify a city and a mascot.)} 

\item \textcolor{Turquoise}{Any method that gets its hands on a
\texttt{Ballplayer} can find out his/her age. But only methods of the
\texttt{Ballplayer} class itself can change its age.}

\item \textcolor{Red}{Every \texttt{Team} object will have a private instance
variable\footnote{Notice that the ``\textbf{--}" immediately before the word
``roster" is a visibility modifier, indicating that the inst var that results
from this association will be private.} called \texttt{roster} which holds a collection (perhaps an
\texttt{ArrayList}) of \texttt{Ballplayer} objects. Each of those
\texttt{Ballplayer}s belongs to only a single \texttt{Team} object, but is not
aware of which \texttt{Team} object that is (\textit{i.e.},
\texttt{Ballplayer} objects don't have an inst var of type \texttt{Team})}.

\item \textcolor{darkblue}{There is a single integer variable \texttt{numTeams}
which is shared among all objects of type \texttt{Team}. It is not visible to
any other class.}

\item \textcolor{Purple}{Somewhere in the static \texttt{main()} method of the
\texttt{Simulator} class we would expect to find code like this: ``\texttt{new
Ballplayer(someName, someAge)}".}

\item \textcolor{Brown}{A \texttt{Simulator} holds on to some number of
\texttt{Team} objects, probably in an instance variable, and each of those
\texttt{Team}s belong only to it.}

\end{enumerate}

Did you pick all those things out? If so, you can read a blueprint, and I
foresee many beautiful buildings in your future!


\chapter{Life in Singleton, USA}

``Singleton" always sounded to me like the name of some small American town,
maybe one where nobody ever gets married. But it's actually the name of our
first (and easiest) \textit{design pattern}, and the subject of this chapter.

\section{Design patterns}

You know how when you sit down to write some code, there are times when you
think, ``wait, I've written this before"? Programmer's d\'{e}j\`{a} vu is
commonplace, especially because certain tips and tricks end up working in
a lot of different settings. For instance, we've all seen this sort of thing:

\vspace{-.1in}
\begin{Verbatim}[fontsize=\small]
    int multiplyAllTogether(int arr[]) {
        product = 1;
        for (int i=0; i<arr.length; i++) {
            product = product * arr[i];
        }
        return product;
    }
\end{Verbatim}

and this:

\pagebreak
\begin{Verbatim}[fontsize=\small]
    boolean hasAtLeastOneFive(int arr[]) {
        for (int i=0; i<arr.length; i++) {
            if (arr[i] == 5) {
                return true;
            }
        }
        return false;
    }
\end{Verbatim}

and this:

\begin{samepage}
\begin{Verbatim}[fontsize=\small]
    int getIndexOfHighest(int arr[]) {
        highestSoFar = arr[0];
        index = 0;
        for (int i=1; i<arr.length; i++) {
            if (arr[i] > highestSoFar) {
                index = i;
                highestSoFar = arr[i];
            }
        }
        return index;
    }
\end{Verbatim}
\end{samepage}

I could go on and on. You might think of these as ``programming patterns."
They're bite-sized, go-to solutions that can handle a myriad of common little
programming scenarios.

It turns out that the same is true of design. Certain motifs in how classes
collaborate with each other crop up again and again in different settings.
They're important enough that they've been identified, described, and named.

The people who first promoted the idea of \textbf{design patterns} were Erich
Gamma, John Vlissides, Ralph Johnson, and Richard Helm, who were thereafter
nicknamed ``The Gang of Four." (You'll hear many references in the software
development industry to a ``Gang of Four design pattern," sometimes
abbreviated ``GoF design pattern." This means one of the 21 named patterns
that appeared in their hugely influential 1994 book \textit{Design Patterns:
Elements of Reusable Object-Oriented Software}. It's a book highly worth
obtaining and reading.)

One thing that's great about this is that just by mentioning one of these
agreed-upon pattern names -- like ``Observer," ``Iterator," or ``Strategy" --
every developer worth their salt will instantly conjure up in their mind the
mechanics of that particular pattern, and know immediately what kind of
problem it's intended to solve. It saves a lot of words trying to describe an
idea you know your fellow developer has seen before, if only you could get
them to realize what you're talking about.

In this brief chapter, we'll cover the simplest GoF pattern of all:
the \textbf{Singleton} pattern.

\section{The Singleton pattern}

The Singleton pattern is so simple it almost doesn't even deserve to be called
a pattern. But it is. And it's easy to figure out when it applies to your
situation: \textbf{a Singleton is used when you have a class for which you
only ever want to instantiate one object.}

If you think about it, this kind of situation is pretty rare. Clearly any
relevant program is going to need to instantiate lots of different
\texttt{Car} objects, or \texttt{Ballplayer}s, or \texttt{Professor}s. There
are occasions, though, when your class isn't so much a category as it is a
special, one-of-a-kind object. Here are some examples:

\begin{itemize}
\itemsep.1em

\item Part of your operating system may have a \texttt{PrinterManager} class
that controls sending documents to various printers. The code will create many 
\texttt{Printer} objects, and many \texttt{Document}s, but only one
\texttt{PrinterManager} which runs traffic control and routes print jobs to
available printers.

\item Your website that collects information about classic rock 'n' roll
albums may have a \texttt{Database} class that represents the underlying data
storage. You might get multiple \texttt{Connection}s to it and instantiate
multiple \texttt{Query} objects to search it, but there's just one
\texttt{Database} as a point of contact.

\item Most programs have some way of configuring them, usually by tweaking the
values of various configuration variables. Your program could have a
\texttt{Config\-uration} class from which the other software components can
fetch the values of the settings as needed. There needs to be only
\textit{one} \texttt{Configuration} object, since they will all share access
to a common set of settings.

\end{itemize}

The Singleton pattern does two things: (1) it ensures that only one
instantiation is possible, and (2) it provides a global point of access to
that one object, so that the rest of the code can get to it.

\section{Implementation}

Here's what a properly-coded Singleton pattern looks like. We'll use the
\texttt{Configuration} example from above:

\begin{Verbatim}[fontsize=\small,samepage=true,frame=single]
class Configuration {
    private static Configuration theInstance;
    
    public static synchronized Configuration instance() {
        if (theInstance == null) {
            theInstance = new Configuration(...);
        }
        return theInstance;
    }

    private Configuration(...) {
        ...
    }

    // The actual methods of the object. For Configuration, this
    //   might include something like:
    public String getParamSetting(String param) {
        ...
    }
}
\end{Verbatim}

Let's go through each part carefully. First, we have a
\textit{\texttt{static}} variable called ``\texttt{theInstance}". Recall that
\texttt{static} here means ``goes with the class as a whole, rather than with
each individual object." The reason for this is \textit{the class itself will
be holding on to its one-and-only instantiated object.} This is one of the few
places we'll be using \texttt{static} stuff in this book, because we need to.
If \texttt{theInstance} were \textit{not} \texttt{static}, then the only way
to get a hold of \texttt{theInstance} would be to have an instance of
\texttt{Configuration} in the first place...which would defeat the purpose of
the pattern.

Note that \texttt{theInstance} is also marked \texttt{private}. This is
partially because of our rule ``all inst vars should always be private,
period," but also because making this variable accessible outside the class
would make the whole pattern collapse. Parts of the code that needed access to
the \texttt{Configuration} singleton instance would try to grab
\texttt{theInstance} and use it, but it might not have even been set to
anything yet!

Next, we have the \texttt{instance()} method. This method is also
\texttt{static}, so that it can be called on the \textit{class} rather than on
an object. And what does it return? A \texttt{Configuration} object...or
perhaps I should say, \textit{the} \texttt{Configuration} object since there's
only ever going to be one.

Unlike \texttt{theInstance}, \texttt{instance()} is public. (Package-level
visibility is also an appropriate choice, depending on how wide your intended
users of this Singleton are.) This is part of the public interface of the
class, designed to be called by code external to the class. The other word on
the declaration line is probably foreign to you: it's called
\texttt{synchronized}, and its purpose is beyond the scope of this chapter.
Very briefly, ``\texttt{synchronized}" prevents two different \textbf{threads}
of execution from entering the \texttt{instance()} method at the same time.
A \textbf{multithreaded program} is one that executes more than one flow of
control simultaneously, each with its own stack. It turns out that if more
than one thread was inside this method at the same time, we might accidentally
instantiate \textit{two} (or more) \texttt{Configuration} objects. For our
single-threaded programs this isn't an issue, but it's good practice to get in
the habit of making your Singleton \texttt{instance()} methods synchronized.

Now let's dive in to the code for \texttt{instance()}. It's very simple, as
you'll see: all it does is say ``if this is the first time anyone's ever
called me, go ahead and instantiate an instance of me, and remember it (in the
\texttt{theInstance} class variable). Then, return the one-and-only instance
of me to the caller, to use to their heart's content."

This is called \textbf{lazy instantiation}: the only thing that will trigger
the one-and-only \texttt{Configuration} object being instantiated is the first
time any other part of the program calls \texttt{Configuration.instance()}. If
nobody ever does call it, then there won't ever be even one instance of this
class created. But assuming someone does, a new \texttt{Configuration} object
will be instantiated \textit{this time only}. From that point on, all the
subsequent times \texttt{Configuration.instance()} is called, that same object
will be returned.

Then we have the constructor. It can do anything that any constructor can do,
which varies widely depending on what kind of class this is. (For the
\texttt{Configuration} example, perhaps it looks at the filesystem for a
\texttt{.config} file, and if it exists, loads it and remembers all its
contents in instance variables.) The important point to emphasize here is that
\textit{the constructor must be \texttt{private}}). That's because if it
weren't private, any old schmo could just write ``\texttt{new
Configuration()}" and get a \textit{second} instance of the class, which is
precisely what we want to avoid. Making the constructor \texttt{private} means
nobody is allowed to instantiate a \texttt{Configuration} object...except for
the \texttt{Configuration} class itself, which we saw in the
\texttt{instance()} method.


\section{Using the Singleton}

This pattern allows any other part of the code base to do things like:

\begin{Verbatim}[fontsize=\footnotesize,samepage=true,frame=single]
  String bgcolor = Configuration.instance().getParamSetting("backgroundColor");
\end{Verbatim}

Whenever we say ``\texttt{Configuration.instance()}" we get back a
\texttt{Configuration} object. (Whether we realize it or not, it's the only
such object that will ever exist.) There's no need to set this to anything, or
to use the word \texttt{new}; we just say ``\texttt{Configuration.instance()}"
every time we need it.

Other than this scaffolding, the rest of your Singleton class can do anything
it wants. It will almost certainly have other (non-static) instance variables,
and other methods to carry out its evil deeds. The ``Singleton part" is just
the \texttt{instance()} and \texttt{theInstance} members, together with the
private constructor.

Singleton is often used in conjunction with the Factory pattern, by the way,
which we will look at in a future chapter.

That's it. Told you it was easy! $\smiley$


\backmatter
\printindex

\end{document}
