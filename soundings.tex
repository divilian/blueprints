
\chapter{Soundings: UML Sequence diagrams}

Class diagrams are the bread-and-butter of UML. They depict the static
features of software systems: the classes, methods, and associations between
them. Complementary to class diagrams are another type of UML artifact called
\textbf{sequence diagrams}. They show the \textit{dynamic} interrelations
between objects as a system's code executes.

Each sequence diagram depicts one scenario, or flow through the system. Unlike
a class diagram, which is sort of ``always true" and shows all the permanent,
reliable features of the program in question, every sequence diagram shows its
own path: its own thread of execution in a particular, hypothetical scenario.
After all, nearly every time you run a program, something different happens,
either because the user makes different choices, network and system latencies
cause various tasks to end at different times, or a random number generator is
involved. A sequence diagram selects just \textit{one} possible outcome and
highlights it start to finish so that an example of how the classes are
intended to interact is unveiled.

I think of a sequence diagram as a ``sounding" in the nautical sense. In
ancient times, ship captains who suspected they were approaching land would
test how deep the water was by probing it with a sounding line. Modern ships
do the equivalent with sonar. A sounding is an exploratory investigation down
one possible path through the water's depths to see what's below. One sounding
doesn't tell you everything about the whole region's topography, but it tells
you a great deal about the specific area you're in. And if you perform several
soundings, you can combine the clues you obtain from each one to build a
mental picture of a wider section of the ocean floor. A programmer can do the
same by learning from several different sequence diagrams.

Sequence diagrams are designed to be perused in conjunction with their
corresponding class diagrams. I always tell students: ``when you look at a
sequence diagram, only look at it with one eyeball; keep the other eyeball on
the class diagram." As we'll see, both diagrams have to be ``in sync" with
each other, since information presented on one must be compatible with what's
on the other.
