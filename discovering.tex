
\chapter{The art of software design}

This chapter marks a watershed of sorts. Up to this point, we've been doing
\textbf{analysis} instead of \textbf{synthesis}. Analysis is when you look at
something that already exists -- a design diagram or a code snippet, say --
and seek to understand it, usually by breaking it down into its constituent
parts. Synthesis, on the other hand, is designing something that doesn't
already exist. Instead of scrutinizing a UML diagram, we're creating a UML
diagram; instead of examining a method, we're writing a method.

Until now, I've presented you with example after example of classes and
methods already written, and diagrams illustrating their various parts. But
now it's time to ask the question: ``how do we figure out what the right
classes and methods \textit{are} in the first place?" It's all well and good
for someone to hand us \texttt{Ballplayer}, \texttt{Team}, and
\texttt{Simulator} classes. But how did we know to create those particular
classes? Why not \texttt{Pitch}, \texttt{Catch}, and \texttt{Hit}? Why not
\texttt{FirstBaseman}, \texttt{Shortstop} and \texttt{Outfielder}? Why not
\texttt{NationalLeague} and \texttt{AmericanLeague}?

Going from the general idea of a program to a list of classes is tricky. It's
as much an art as a science. It calls for intuition and imagination more than
adherence to a set of rules. Nevertheless, there are principles that guide the
selection of good classes, and we'll talk about them in this chapter.

Of all the OO pioneers who weighed in on the question of how to arrive at a
good design, the one who had the most influence on me was Rebecca Wirfs-Brock,
who invented the technique called \textbf{responsibility-driven design}. I'm
highly indebted to her, and recommend the original book authored by her and
her colleagues.\footnote{Wirfs-Brock, Wilkerson, Wiener, \textit{Designing
Object-Oriented Software.} Prentice Hall, 1991.}

\section{``Discovering the design"}

I don't know who first coined the phrase ``discovering the design" (it
certainly wasn't me; it might have been Wirfs-Brock) but when I originally
heard it my ears perked up. It sounded strangely paradoxical. ``Design" was
something you brought to the table and imposed on your world, right? Not
something you found already there. ``Design" seemed like a matter of
\textit{invention}, not \textit{discovery}; it was surely something you did to
a steam engine, not to a planet.

Yet hidden in this phrase is a powerful technique for OO design that attempts
to \textit{let the requirements speak for themselves.} One of Rebecca
Wirfs-Brock's great ideas was to begin with a written description of a
software program in action, and to cull from the language clues as to what the
``correct" classes are.

Let me immediately clarify that ``correct" does not mean ``there is one and
only one `right' set of classes" for a particular program. In fact, there are
many such choices, some better than others, some downright awful. What we
mean by ``the correct classes" is a set of classes (and their corresponding
inst vars and methods) that will:

\begin{compactitem}
\item represent the domain well
\item work seamlessly together
\item be amenable to adaptation as the system requirements evolve
\item distribute the responsibilities evenly among several classes
\item neither duplicate nor omit important functionality
\end{compactitem}

You get the picture. A good design is elegant, flexible, maintainable, and
robust to change. Many choices of classes will not meet these goals. A few
will. ``The correct set of classes" means \textit{any} set of classes that
will do so reasonably well.

\section{Straight from the horse's mouth}

Wirfs-Brock's procedure is paraphrased in Figure~\ref{fig:discovering}. We
have to somehow come up with a written description to kick things off. Often,
a \textbf{requirements document} has been authored by someone higher-up on our
company's food chain, and can be mined for much gold. Sometimes, we ourselves
take a step back and bang out a few paragraphs that describe what users do and
experience as they work with the system.

\setlength{\fboxsep}{10pt}
\begin{figure}
\centering
\fbox{\parbox{.85\textwidth}{
Start with a written description of the software, and:
\begin{compactenum}
\itemsep.03in
\item Identify all noun phrases.
\item Eliminate obviously bad ones:
    \begin{compactenum}
    \item probable duplicates
    \item nouns that aren't instantiate-able
    \item things you obviously wouldn't represent
    \item likely \textit{attributes} of a class, not classes themselves
    \end{compactenum}
\item See which of the remaining ones ``feel right." Identify what each one
\textbf{knows} and can \textbf{do}. These are your \textbf{candidate classes}.

\end{compactenum}
}}
\vspace{.1in}
\caption{Procedure for ``discovering the design."}
\label{fig:discovering}
\end{figure}

The essential point is that the requirements themselves speak loudly about
what classes would be appropriate for the program it describes. Let's see how.

\subsection{Nouns, and only nouns}

If you flash back to \textit{Schoolhouse Rock} or \textit{Sesame Street},
you'll remember your grammatical parts of speech and realize that a
\textbf{noun} is the right kind of word for a class name. Every object (and
therefore the class it's an instance of) is a ``person, place, or thing," not
an action word, modifier, or anything else.

Further, not just not any old noun will do. Consider the list in
Figure~\ref{fig:nouns}: these are all nouns, but only \textit{one} makes a
valid class name. Can you find it?

\setlength{\fboxsep}{1pt}
\begin{figure}[hb]
\centering
\fbox{\parbox{.85\textwidth}{
\begin{multicols}{3}
\begin{compactitem}
\renewcommand\labelitemi{\freakingtilde}
\item happiness
\item Justin Bieber
\item oxygen
\columnbreak
\item crocodile
\item teamwork
\item width
\columnbreak
\item communism
\item recreation
\item London
\end{compactitem}
\end{multicols}
}}
\vspace{.1in}
\caption{All nouns...but not all good class names.}
\label{fig:nouns}
\end{figure}

I claim the only legit class name in this list is
\underline{\textit{crocodile}}. Here's why. First, some of these entries are
``proper nouns" which means they refer to specific instances of things, rather
than categories. In English, we almost always use \textit{capital letters} to
denote proper nouns, which means when you see ``Justin Bieber" and ``London"
you can immediately roll your eyes and move on.

Second, most of the other nouns aren't \textit{instantiate-able}. Here's the
litmus test for whether a noun is instantiate-able: can you meaningfully
put the word ``a" (or ``an") before it? And can you meaningfully make it
plural and put a number (like thirteen) before it?

Clearly not. All of these phrases are plainly ridiculous:

\begin{multicols}{2}
\begin{compactitem}
\item four happinesses (?)
\item eleven oxygens\footnote{One could imagine a chemical analysis program
that dealt with oxygen atoms, among other things, and I've heard chemists
speak loosely of things like ``an extra oxygen" or say ``that molecule has
five oxygens." I still like \texttt{OxygenAtom} much, much better as a class
name even here, though.} (?)
\item a teamwork (?)
\columnbreak
\item three communisms (?)
\item a communism (?)
\item nineteen recreations (?)
\end{compactitem}
\end{multicols}

Remember, the only thing we ever really do to a class is make instances of it,
to which we can do things. If you can't imagine a ``\texttt{new Communism()}"
or ``an \texttt{ArrayList} of \texttt{Happiness}es," it has no business being
a class.

The closest contender to crocodile is the word \textit{width}. There may be
cases where this is a sensible class, but the reason I discard it is that a
width is almost certainly a \textit{modifier} of some other object, rather
than an object itself. One could imagine \texttt{Building}, \texttt{Image},
and \texttt{Rectangle} objects that all had an instance variable called
\texttt{width}; it's harder to imagine ``a width" as an entity in its own
right, with its own properties and operations.

\subsubsection{Noun phrases}

By the way, it's often the case that instead of a bare noun, we use a
\textbf{noun phrase} as a class name. A noun phrase is simply a noun with one
or more modifiers. ``Grizzly bear," ``chess tournament," and ``public liberal
arts college" are examples.

\subsubsection{Singular, not plural}

Finally, it should hardly be worth stating that all class names must be
\textbf{singular}, not plural. I don't work in ``a buildings," but a
\textit{building}; and nobody has ``a dogs" as a pet. When we instantiate an
object, we're going to say ``\texttt{Crocodile alice = new Crocodile()}", not 
``\texttt{Crocodiles alice = new Crocodiles()}".

\subsection{Carrying out the process}

\textbf{1. Identify all noun phrases.} Okay. We begin our semi-automated
process of deriving class names by starting with a written description of the
program's requirements. Here's a short example:

\setlength{\fboxsep}{10pt}
\begin{center}
\large
\fbox{\parbox{.85\textwidth}{
\textsf{A bicycle store needs to manage its inventory. Shipments of various
models of bicycles are received every week from its suppliers, and customers
place individual orders for bikes and other accessories from the store. The
store manager must be able to place orders from vendors, maintain
contact information so they can be confirmed or canceled, and view lists of
the incoming products and their expected arrival dates. The manager also must
be able to record multi-item orders from individual customers, accept and
record down payments, and track inventory levels to ensure that enough items
are ordered to satisfy customer demand.}}}
\end{center}

Rebecca Wirfs-Brock's process from Figure~\ref{fig:discovering} calls for
sifting through the requirements description and \textit{circling all the noun
phrases}. Unless it's an exact duplicate of one that previously occurred, be
conservative and circle every one. It would be a good exercise for you to do
this yourself in the box above, and then compare with my answer:

\setlength{\fboxsep}{10pt}
\begin{center}
\large
\fbox{\parbox{.85\textwidth}{
\textsf{A \bluebox{bicycle store} needs to manage its \bluebox{inventory.}
\bluebox{Shipments} of various
\bluebox{models} of \bluebox{bicycles} are received every \bluebox{week} from its
\bluebox{suppliers}, and \bluebox{customers}
place \bluebox{individual orders} for \bluebox{bikes} and other
\bluebox{accessories} from the \bluebox{store}. The
\bluebox{store manager} must be able to place \bluebox{orders} from
\bluebox{vendors}, maintain
\bluebox{contact information} so they can be confirmed or canceled, and view
\bluebox{lists} of
the \bluebox{incoming products} and their \bluebox{expected arrival dates}. The
\bluebox{manager} also must
be able to record \bluebox{multi-item orders} from \bluebox{individual
customers}, accept and
record \bluebox{down payments}, and track \bluebox{inventory levels} to ensure
that enough \bluebox{items}
are ordered to satisfy \bluebox{customer demand}.}}}
\end{center}

This is the raw material for the rest of the process. If we make everything
singular and lower-case, this leaves us with the following list:

\begin{samepage}
\begin{multicols}{3}
\small
\begin{compactitem}
\renewcommand\labelitemi{\raisebox{0.25ex}{\tiny$\bullet$}}
\item bicycle store
\item inventory
\item shipment
\item model
\item bicycle
\item week
\item supplier
\item customer
\item individual order
\columnbreak
\item bike
\item accessory
\item store
\item store manager
\item order
\item vendor
\item contact information
\item list
\item incoming product
\columnbreak
\item expected arrival date
\item manager
\item multi-item order
\item individual customer
\item down payment
\item inventory level
\item item
\item customer demand
\end{compactitem}
\end{multicols}
\end{samepage}

\textbf{2a. Eliminate probable duplicates.} According to
Figure~\ref{fig:discovering}, the next step is to eliminate likely duplicates.
Obviously things like ``bicycle" and ``bike" refer to the same conceptual
entity, of course; we're hardly going to have a \texttt{Bicycle} class and a
separate \texttt{Bike} class in our program!

This isn't always 100\% straightforward, but it's usually 99\% so. Different
synonyms and turns of phrase are pretty easy to detect. I think we can be
pretty safe boiling this list down into a slightly smaller one, where
duplicates are shown:

\begin{samepage}
\begin{multicols}{2}
\small
\begin{compactitem}
\renewcommand\labelitemi{\raisebox{0.25ex}{\small$\bullet$}}
\item bicycle store == store
\item inventory
\item shipment
\item model
\item bicycle == bike
\item week
\item supplier == vendor
\item customer == individual customer
\item individual order == order == multi-item order
\columnbreak
\item accessory
\item store manager == manager
\item contact information
\item list
\item incoming product
\item expected arrival date
\item down payment
\item inventory level
\item item
\item customer demand
\end{compactitem}
\end{multicols}
\end{samepage}

The choice of which synonym to retain is mostly aesthetic. All other things
being equal, I usually choose the shorter one.

\vspace{.1in}
\textbf{2b. Eliminate nouns that aren't instantiate-able.} Now we apply our
test: ``can we put `a/an' or a number before the noun phrase, and have it make
sense?"

Actually almost all of these remaining entries pass that test, with the
exception of \textit{inventory}, \textit{inventory level}, and
\textit{customer demand}, and possibly \textit{contact information}. While one
could indeed envision ``two different inventories," ``four inventory levels,"
and ``five customer demands" in other contexts, it's pretty clear from the
text that these are being used as abstract concepts, not individual objects.
``Contact information" is a closer call, but by inspecting the requirements
again, we can see that this is really an attribute of
\textit{vendor/supplier}. We'll therefore strike the idea of a
``\texttt{ContactInformation}" class. We're now down to:

\begin{samepage}
\begin{multicols}{3}
\small
\begin{compactitem}
\item store
\item shipment
\item model
\item bicycle
\item week
\columnbreak
\item supplier
\item customer
\item order
\item accessory
\item manager
\columnbreak
\item list
\item incoming product
\item expected arrival date
\item down payment
\item item
\end{compactitem}
\end{multicols}
\end{samepage}

\vspace{.1in}

\textbf{2c. Eliminate things you obviously wouldn't represent.} When you look
at some of these surviving noun phrases, you scratch your head. Would we
really have a ``\texttt{Week}" class? Surely not. Also, although this program
is no doubt intended for the manager of a store, does it really make sense to
represent the \texttt{Manager} and \texttt{Store} as objects? We'll cross out
all three of these.

\vspace{.1in}

\textbf{2d. Eliminate likely \textit{attributes} of a class, not classes
themselves.} Now things are getting somewhat subjective, but some of these
remaining nouns seem ``too small" to be their own class. Consider
\textit{expected arrival date}. Surely this is better modeled as a property of
an order, rather than its own individual object. The same could be said for
\textit{down payment}. Generally speaking, noun phrases that seem to refer to
bits of data that have an obvious ``home" in another class ought to be modeled
as inst vars, not classes.

So now all that remains are:

\vspace{-.3in}
\begin{samepage}
\begin{center}
\parbox{.8\textwidth}{
\begin{multicols}{2}
\begin{compactitem}
\item shipment
\item model
\item bicycle
\item supplier
\item customer
\columnbreak
\item order
\item accessory
\item list
\item incoming product
\item item
\end{compactitem}
\end{multicols}
}
\end{center}
\end{samepage}

\textbf{3. See which of the remaining ones ``feel right."}


