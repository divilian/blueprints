
\chapter{Documenting an API}

``\textbf{API}" -- which historically stands for ``Application Programming
Interface" -- is one of the dumber acronyms you'll encounter. And worse, it's
commonly used to mean two different things: (1) a set of classes (and their
methods) which a programmer could make use of in their own code, and (2) the
documentation describing those classes/methods.\footnote{By the way, the term
``API" isn't used only for object-oriented software. One could write some
old-school procedural code (with functions and data structures, rather than
encapsulated classes and methods), describe it, and call that an API as well.}

In common lingo, people speak of ``programming to an API," which means
``writing some code which conforms to those documented classes/methods." Every
time you've used an \texttt{ArrayList} or a \texttt{Scanner}, in fact, you
have been doing this. Instantiating such objects, calling methods on them, and
(importantly) reading the documentation at
\url{https://docs.oracle.com/javase/8/docs/api/} to find out how they
operate is all part of leveraging the built-in Java API for your own purposes.

These days, when people talk about using an API, they often mean writing code
that connects over the Internet to some publicly-available service or database
of information. Nearly every major Internet player these days -- Google,
Youtube, Instagram, Flickr, eBay, Twitter, Dropbox, Spotify, Amazon,
\texttt{data.gov}, GeoDB cities, \textit{etc.} -- has a publicly-accessible
API. This allows you to write code (in any language) to connect to it and
query it for information, perform commands, make purchases, and so forth.
Browse \url{dev.twitter.com} to get an idea of the rich functionality
available to anyone with the technical savvy to understand and exploit an API.

It's an interconnected, collaborative world. Developers rarely write all the
code themselves anymore, all on an isolated island. Instead, they share code
and for others to use, and take advantage of what's been shared with them. If
you can figure out how to effectively do that, you've increased your
programming potential a hundredfold.

\section{The importance of good documentation}

Now in order to make it possible for other developers to use the code you so
painstakingly wrote, it must be \textbf{documented} in a way that is clear,
complete, and unambiguous. To appreciate the importance of this, I want to
lead you in a thought experiment.

First, pretend you're back in the 1990's, a glorious time to be young and
alive. In particular, pretend that \textit{GPS and cell phones are not yet
commonly available.} (Believe it or not, this was true in the recent past.)

Let's suppose it's Friday night, and you're going to a party at the apartment
of your acquaintance Biff. Biff lives up in North Stafford, and you've never
been to his place before. Luckily, your close friend Filbert is also going to
the party, and he's been to Biff's on many occasions. You're picking him up at
8pm.

Consider the following two scenarios.

\begin{description}

\item[Scenario A:] You'll pick up Filbert (and possibly one or two others), and
drive together to Biff's apartment.

\item[Scenario B:] Filbert calls you at the last minute and says that he's
getting a ride with somebody else. He gives you \textit{written directions} to
Biff's, however, so that you can get to the party on your own.

\end{description}

My question: in which of the above two scenarios are you \textit{more} likely
to successfully arrive at the party without getting lost? Or are both cases
equally likely?

The careless thinker might at first conclude that the two cases are equally
likely. After all, they both depend on Filbert's knowledge of how to get to
Biff's. In one case, Filbert's verbalizing the directions as you drive, and in
the other case, he's laying them all out for you in advance. But
theoretically, as long as Filbert knows how to get there, you'll be successful
in both scenarios.

Theoretically. But in the real world, as everyone knows, it usually doesn't
work like that. In scenario A, and Filbert in the passenger seat, you have the
chance to interactively ask about every intersection and every turn. In
scenario B, \textit{Filbert had to specify everything perfectly in advance.}
He had to describe the route with no errors, since there would be no chance to
make corrections en route. He had to anticipate every question you might have,
since he wouldn't be there to answer them. That's a lot of pressure on Filbert
to give good directions.

Consider the following very realistic possibilities:

\begin{itemize}
\itemsep.1em

\item Filbert wrote ``left" when he meant ``right" in step 3 of the directions
because he's human.

\item Filbert just plain forgot step 5 of the directions because he's human.

\item When he wrote, ``turn left at the next opportunity," he meant ``at the
next intersection," and assumed that would be obvious to you. However, you
quite naturally thought he meant ``the very next possible left," which was
down a side road.

\item A road is closed, or there's a traffic jam, and you need to go around it
in order to make it to the party on time.

\item \textit{Etc.}

\end{itemize}

You can think of a dozen more. In all these cases, having Filbert with you in
the car allows you to clarify ambiguities, fill in omissions, ask questions as
they arise, and change course in response to unexpected circumstances. With
the written directions, you have none of those options. Put another way,
Filbert isn't even at your disposal in Scenario B: your only asset is
Filbert's brain dump, as he was conceiving it at 6:13pm.

And by the way: most people are pretty bad at giving directions.

\subsection{Collaborating with someone you'll never meet}

In case the above analogy isn't plain, Scenario A corresponds to a software
development team where your teammates are just down the hall. They're just an
email or a Slack away. You can ask questions, report bugs, or even request
alterations as the need arises. The pressure is off, as far as documentation
is concerned. In fact, why even bother trying to document everything
exhaustively in advance, if your teammates can ask focused questions in real
time?

Scenario B corresponds to you using a public API. The instructions written by
a developer you will never meet are \textit{your one and only chance} to
comprehend how to use the thing. Those instructions had better be darned good,
because there is no chance to ask questions on the road. They'd better clearly
and exhaustively contain \textit{everything} you're likely to want to know.
