
\chapter{UML class diagrams}
\label{ch:classDiags}

\index{dynamic (view of memory)}
\index{static (view of memory)}
We spent last chapter discussing the \textbf{dynamic} view of a program: what
happens to memory, step by step, as it unfolds. In this chapter, we'll switch
to a \textbf{static} view: long-term, what are the program's classes, methods,
and relationships between them?\footnote{The words ``dynamic'' and ``static''
are ubiquitous in computer science, and mean a zillion different unrelated
things. For example, we've already seen the Java ``\texttt{static}'' keyword,
and how it indicates class-level rather than an object-level ownership. We've
also hinted at the stack having ``statically-allocated memory'' and the heap
being ``dynamically-allocated.'' These terms are \textbf{\textit{unrelated}} to
our use of the words in this chapter. At present, by ``dynamic'' we mean ``the
contents of memory changing as the program runs''; and by ``static'' we mean
``the consistent, permanent characteristics of a program, quite apart from how
it might be behaving at any moment, which include its classes, methods, and
associations.''}

\index{class diagram}
\index{blueprint}
If there's a type of UML diagram that deserves the name ``blueprint,'' it's the
\textbf{class diagram}. Class diagrams depict a high-level, stable perspective
of a software system. When you want to figure out how a large OO program
works, or when you're tasked with implementing a system that someone else has
designed, the first thing you look at are its class diagram(s).

UML class diagrams contain a number of elements, each of which has a very
specific meaning. We'll cover each in turn.

\section{Classes}

\index{class@\texttt{class}}
Unlike memory diagrams, which depict objects, class diagrams contain classes
(duh). We've already seen what a single class looks like in
section~\ref{sec:UMLclasses} (\textit{e.g.}, the left side of
Figure~\ref{fig:classObject}.) Most class diagrams contain many such classes.
Recall that each class has three compartments, containing the class's name,
its inst vars, and its methods, in that order.

\index{car@\texttt{Car}}
By the way, one flexible (yet slightly annoying IMO) aspect of UML is that it
allows \textbf{varying levels of detail}. In other words, on a particular
diagram, you may or may not want to show all the instance variables and
methods, because it may or may not be relevant to the purpose of that
particular diagram. Similarly, you may or may not want to show all the aspects
of each inst var or method; perhaps it's too early in the design process to
completely specify all the parameters and return types, for example. To
illustrate, all three pictures in Figure~\ref{fig:graceful} are legit
ways of representing the \texttt{Car} class. We can include as much or as
little detail as we please.

\begin{figure}[ht]
\centering
\includegraphics[width=1\textwidth]{graceful.pdf}   % 670x320
\caption{Three equally valid ways to draw the \texttt{Car} class on a class
diagram, depending on how much detail it makes sense to include.}
\label{fig:graceful}
\end{figure}

The reason I find this annoying, by the way, is that it's ambiguous. If you
see no inst vars in the second box, does that mean (a) that class \textit{has}
no inst vars, or (b) the designer didn't think it was relevant to include them
on this particular diagram? No way to really know.

\section{Associations}

\index{association}
Perhaps the most important bits of information on a class diagram are the
\textbf{association}s between classes. An association means that two classes
collaborate together in some way to achieve some larger purpose. It is
indicated on a class diagram by a line connecting the two classes. Different
types of lines represent different kinds of relationships between the classes.
It's important not to mix them up, because if you do, you're dictating
something incorrect to the programming team about how the classes are intended
to work.

\begin{figure}[ht]
\centering
\includegraphics[width=0.7\textwidth]{assocArrows.pdf}   % 350x200
\vspace{.2in}
\caption{Diagrammatic elements for different association types.}
\label{fig:assocArrows}
\end{figure}

\subsection{Dependency associations}

\index{dependency!dependency association}
\index{line!dashed (-\ -\ -)}
\index{arrowhead (\textbf{>})}
Figure~\ref{fig:assocArrows} shows some of the UML association arrows and
their meaning. (There are others we'll get to in future chapters.) The dashed
line with a crow's foot arrowhead is called a \textbf{dependency}, and is the
``weakest'' of the association types. When I say weak, I mean that the
relationship between the two classes isn't as important, nor as permanent, as
with the other association types we'll discuss later.

A dependency between classes \texttt{A} and \texttt{B} can be thought of in a
couple of ways:

\begin{compactitem}
\item One or more methods of the \texttt{A} class will \textit{call methods
on} a \texttt{B} object.
\item The \texttt{A} class \textit{is dependent on the interface of} the
\texttt{B} class.
\end{compactitem}

\index{public interface}

The word \textbf{interface} -- like stack, heap, dynamic, static, and many
other computer science words -- has multiple meanings. We've seen one usage
already, in Figure~\ref{fig:mostImportant} (p.~\pageref{fig:mostImportant}).
We'll talk about the Java \texttt{interface} keyword later in the book. For
now, when I say interface I mean \textit{those aspects of a class that a user
of that type of object can see.} This boils down to: the methods you can call
on it, together with their argument lists and return types. Specifically, the
interface does \textit{not} include the method implementations (the bodies of
the methods), nor the instance variables.

\index{scissorKick@\texttt{.scissorKick()}}
If you think about it, you'll realize why the above two bullet points are
actually equivalent. Suppose some class \texttt{A} method has this line of
code in it: ``\texttt{String s = B.scissorKick(15)}''. Then clearly the code in
the \texttt{A.java} file is \textit{dependent} on the fact that class
\texttt{B} has a \texttt{.scissorKick()} method, and that it takes an integer,
and returns a \texttt{String}. If any of that ever changed in the
\texttt{B.java} file, then class \texttt{A} would be impacted.

\begin{figure}[ht]
\centering
\includegraphics[width=0.9\textwidth]{dependencyAssoc.pdf}   % 
\caption{Examples of dependency associations.}
\label{fig:dependencyExamples}
\end{figure}

\index{stereotype}
\label{stereotype}
The strange-looking words adjacent to the dependency arrows in
Figure~\ref{fig:assocArrows} go by the even stranger-sounding term
\textbf{stereotypes}. A stereotype in UML is an extra bit of information that
enhances part of a diagram (an association arrow, as here, or sometimes a
class, method, or other element) by making its meaning more precise.
Stereotypes are usually displayed enclosed by double-wakkas
(``$\ll$...$\gg$'').

\index{uses@$\ll$uses$\gg$ stereotype}
\index{instantiates@$\ll$instantiates$\gg$ stereotype}
In the case of dependency associations, the stereotype ``$\ll$uses$\gg$'' means
pretty much what a dependency always means: that the designer intends class
\texttt{A} to ``use'' (\textit{i.e.}, get its hands on, and call method(s) on)
object(s) of class \texttt{B}. The ``$\ll$instantiates$\gg$'' stereotype goes a
bit further, and implies that some method of \texttt{A} will
\textit{instantiate} \texttt{B} objects in addition to merely calling methods
on them.

\index{Dungeons \& Dragons}
\index{battle@\texttt{Battle}}
\index{die@\texttt{Die}}
The examples in Figure~\ref{fig:dependencyExamples} are from a Dungeons \&
Dragons type combat simulator. A \texttt{Battle} object represents a fight
between adventurers and monsters. While simulating this fight, a
\texttt{Battle} will make use of one or more \texttt{Die} (singular of ``dice'')
objects to roll random numbers that determine the outcome. This is a
``$\ll$uses$\gg$'' association, since \texttt{Battle}'s code now depends on
\texttt{Die}'s interface not changing.

\index{wizard@\texttt{Wizard}}
\index{ranged spell@\texttt{RangedSpell}}
Elsewhere in the program, wizards sometimes cast ranged spells, like fireballs
or lightning bolts, to damage distant enemies. So in the simulator, a
\texttt{Wizard} object might instantiate a \texttt{RangedSpell}
object to carry out this attack. Since somewhere in the \texttt{Wizard} class's
code there will be a ``\texttt{new RangedSpell()}'' line, we say that
\texttt{Wizard} $\ll$instantiates$\gg$ \texttt{RangedSpell}.

\subsubsection{Dependencies in code}

\index{dependency!dependency association}
\index{uses@$\ll$uses$\gg$ stereotype}

Now what would we expect to see in the code that would reflect this kind of
association? In the ``$\ll$uses$\gg$'' case, we expect to see one or more
methods of the \texttt{A} class (\texttt{Battle}, in our example) making method
calls on \texttt{B} (\texttt{Die}) objects. Perhaps something like this:

\begin{Verbatim}[fontsize=\scriptsize,samepage=true,frame=single]
class Battle {
    ...
    void resolveAttack(Adventurer a, Monster m, Die d) {
        ...
        if (d.roll() < a.currentWeapon().attackStat()) {
            ...
        }
    }
    ...
}
\end{Verbatim}

The design diagram doesn't specify exactly which \texttt{Die} method(s) will be
called where, just that method calls are expected. This communicates something
important to the programmer.

\index{instantiates@$\ll$instantiates$\gg$ stereotype}

For ``$\ll$instantiates$\gg$'', we'd expect to see the phrase ``\texttt{new
B()}'' (``\texttt{new RangedSpell()},'' in our example) somewhere in some
method of the \texttt{A} class (\texttt{Wizard}, in our example):

\begin{Verbatim}[fontsize=\scriptsize,samepage=true,frame=single]
class Wizard {
    ...
    void takeAction(ArrayList<Monster> enemies) {
        ...
        if (enemies.size() > 3) {
            RangedSpell fireball = new RangedSpell("Fireball", 60, 12);
            fireball.cast();
            ...
        }
    }
    ...
}
\end{Verbatim}

\subsection{``Has-a'' associations}

\index{hasa association@``has-a'' association}
\index{association!has-a@``has-a''}
\index{line!solid (---)}
\index{arrowhead (\textbf{>})}

The next strongest type of association has a bizarre name: it's called
``\textbf{has-a}.'' We denote it with a solid arrow between classes, with a
crow's foot on one side or both.

\index{instance variable (inst var)}
\index{collection}
When class \texttt{A} has-a class \texttt{B}, that is nearly always a signal
to the programmer that \texttt{A} should have an \textit{instance variable} of
type \texttt{B}.\footnote{Or perhaps a \textbf{collection} of \texttt{B}
objects rather than a single \texttt{B} object, as we'll see later in the
chapter.} In other words, not only does an \texttt{A} object call
methods on a \texttt{B} (as in the dependency association), but an \texttt{A}
object actually holds on to one (or more) \texttt{B} objects for the
long-haul.

\index{pizza@\texttt{Pizza}}
Now in some cases, the ``has-a'' verbiage makes perfect sense. If our Domino's
Pizza delivery manager application had a \texttt{Pizza} class and a
\texttt{Topping} class, it would be no-brainer to say that every
\texttt{Pizza} has-a \texttt{Topping}. It conjures up in our minds a picture
of containment, or ownership. Perfect. However, we also use this type of
association in cases where containment doesn't make sense at all.

For example, in the same application it would be quite sensible to say that
``every \texttt{Pizza} has-a \texttt{DeliveryCar}.'' But obviously the delivery
car isn't ``inside'' the pizza in the same physical way that the toppings are
inside it. So what does it mean then?

\begin{figure}[ht]
\centering
\includegraphics[width=1\textwidth]{wrongRightHasA.pdf}   % 
\caption{The wrong, and right, way to visualize a ``has-a'' association in
Java.}
\label{fig:wrongRightHasA}
\end{figure}

\index{containment}
The key is making sure you have the right mental model.
Figure~\ref{fig:wrongRightHasA} shows both the wrong, and the right, way to
envision a has-a relationship (at least, in Java). In memory, there is
\textit{no} ``containment'' as in the left-hand (wrong) image. The
\texttt{Topping} object isn't enclosed inside the \texttt{Pizza}, or even
exclusively owned by it. It's simply pointed to by one of the \texttt{Pizza}
object's inst vars. The right-hand side of the figure is the correct one --
and I daresay it's not problematic at all to think of a \texttt{Pizza}
``having'' a \texttt{DeliveryCar} in this way. All it really means is that a
\texttt{Pizza} object ``knows about'' a \texttt{DeliveryCar}, which is the
particular car that's delivering it.

\index{navigability}
\index{association!bidirectional}
Another reason that the correct mental model of ``has-a'' is important is that
it is possible, and even common, for the association to go \textit{both ways}.
We use the term \textbf{navigability} for the question ``which direction does
the arrow go -- from \texttt{A} to \texttt{B}, from \texttt{B} to \texttt{A},
or both?'' When it goes both ways, we call it a \textbf{bidirectional}
association.

\begin{figure}[ht]
\centering
\includegraphics[width=1\textwidth]{bidirectional.pdf}   % 750x235
\caption{A bidirectional ``has-a,'' depicted on a class diagram (left) and a
memory diagram.}
\label{fig:bidirectional}
\end{figure}

An example is the left-hand side of Figure~\ref{fig:bidirectional}. Here, our
\texttt{Driver} class and our \texttt{DeliveryCar} class each know about the
other, and in fact each hold on to an instance variable of the other type. If
we viewed this \texttt{A}-having-an-instance-variable-of-type-\texttt{B} thing
as the \texttt{A} object \textit{enclosing} the \texttt{B}, we'd blow a fuse.
\texttt{A} would contain \texttt{B}, which would contain \texttt{A}, which
would contain \texttt{B}, which... That way madness lies. But notice that
nothing paradoxical happens at all in the corresponding memory diagram on the
right-hand side of the figure. Each object points to the other, so that a
\texttt{Driver} object knows which \texttt{DeliveryCar} he/she is driving, and
a \texttt{DeliveryCar} also knows which \texttt{Driver} is driving it. No
biggie.

\begin{figure}[ht]
\centering
\includegraphics[width=1\textwidth]{wrongHasA.pdf}   % 640x180
\caption{One incorrect way to model an instance variable. The ``has-a'' arrow
\textit{already} indicates that every \texttt{Pizza} has-a \texttt{Topping}: the
extraneous \texttt{topping} entry in the \texttt{Pizza} class's second box is
redundant and incorrect.}
\label{fig:wrongHasA}
\end{figure}

Note, by the way, that the has-a arrow implies the existence of the inst var
\textit{all by itself}. The class diagram should \textit{not} contain a
duplicate copy of the inst var in its second compartment. That would be
redundant, and is considered an error (see Figure~\ref{fig:wrongHasA}).

\subsubsection{``Has-a'' associations in code}

\index{hasa association@``has-a'' association}
Obviously instance variables are how ``has-a'' associations are manifested in a
Java program. For \texttt{Pizza} and \texttt{Topping}, we'd see:

\begin{Verbatim}[fontsize=\scriptsize,samepage=true,frame=single]
class Pizza {
    ...
    Topping topping;
    ...
}
\end{Verbatim}

\index{association!bidirectional}
and for our bidirectional \texttt{Driver}/\texttt{DeliveryCar}, we'd see both
\begin{Verbatim}[fontsize=\scriptsize,samepage=true,frame=single]
class Driver {
    ...
    DeliveryCar car;
    ...
}
\end{Verbatim}

and

\begin{Verbatim}[fontsize=\scriptsize,samepage=true,frame=single]
class DeliveryCar {
    ...
    Driver currentDriver;
    ...
}
\end{Verbatim}

These examples both assume that a \texttt{Pizza} has only \textit{one}
\texttt{Topping}, \textit{etc.} If this isn't so, we'd use some kind of
container class instead:

\begin{Verbatim}[fontsize=\scriptsize,samepage=true,frame=single]
class Pizza {
    ...
    ArrayList toppings;
    ...
}
\end{Verbatim}

More on that later.

\subsection{Aggregation associations}

\index{association!aggregation}
\index{aggregation association}
Continuing on towards the ``stronger'' end of the association continuum, an
\textbf{aggregation} implies exclusive ownership of the object(s) in question.
In other words, if \texttt{A} aggregates \texttt{B}, not only does it mean
that \texttt{A} has an instance variable of type \texttt{B}, but that
\textit{no other \texttt{A} object also has that \texttt{B}.}

\index{object}
This is frequently misinterpreted, so let me expand on that. The
``exclusivity'' thing is a statement about \textit{objects}, not classes. If
\texttt{A} aggregates \texttt{B}, that does \textit{not} mean that no other
class can have an instance variable of type \texttt{B}. Rather, it means that
if a particular \texttt{B} object is pointed to by an \texttt{A} object, no
\textit{other} \texttt{A} object also points to that \texttt{B}.

\begin{figure}[ht]
\centering
\includegraphics[width=1\textwidth]{aggregationAssoc.pdf}   % 620x220
\caption{Examples of aggregation associations.}
\label{fig:aggregationAssoc}
\end{figure}

\index{diamond!white ($\lozenge$)}
\index{line!solid (---)}

Examples appear in Figure~\ref{fig:aggregationAssoc}. Note carefully: the
diamond appears \textit{on the ``aggregator'' side} of the arrow;
\textit{i.e.}, adjacent to the class that will have the instance variable. (I
remember getting this backwards at first.)

\index{professor@\texttt{Professor}}
\index{student@\texttt{Student}}
In the first example, for a Banner-like college enrollment management system,
each \texttt{Professor} will teach some number of \texttt{Section}s in a given
semester. If Professor Jones is assigned to teach section 03 of BIOL 121, then
no \textit{other} professor is also assigned to that section. That's what the
white diamond communicates.

\index{photo@\texttt{Photo}}
\index{album@\texttt{Album}}
In the second example, from a Facebook-like social networking site, users can
arrange their \texttt{Photo}s into \texttt{Album}s. As indicated on this
diagram, a given \texttt{Photo} is \textit{not} intended to simultaneously
belong to more than one \texttt{Album}. (If we wanted to relax that
constraint, and permit photos to belong to multiple albums at once, we would
get rid of our white diamond and use a plain-old ``has-a'' arrow instead.)

\subsubsection{Aggregations in code}

\index{hasa association@``has-a'' association}
Aggregation is intended to imply some sort of collection or ownership
relationship between the two classes. However, in terms of the Java code that
you initially write, \textit{there is no immediate difference between an
aggregation and a regular ``has-a.''} In both cases, you'll make an inst var of
the appropriate type in the appropriate place. The code difference between
aggregation and has-a won't come out until later, when the class methods are
being implemented. That white diamond is more of a long-term signal to the
programmer about how two classes are generally intended to operate together,
rather than being a cue to write the first bit of code differently than you
otherwise would.

\subsection{Composition associations}

\index{association!composition}
\index{composition}
\index{lifespan}
\index{dependency!lifespan}
\index{diamond!black ($\blacklozenge$)}
\index{line!solid (---)}
The last association type we'll cover, and the most tightly-binding between
classes, is called \textbf{composition}. It's a lot like aggregation (even the
diamond syntax is the same, except it's black) but with one difference. With
composition, not only does an \texttt{A} object have exclusive ownership over
its \texttt{B} object(s), but there's a \textbf{lifespan dependency} as well:
if the \texttt{A} ever disappears, its constituent \texttt{B}(s) should also
cease to exist.

\begin{figure}[ht]
\centering
\includegraphics[width=1\textwidth]{compositionAssoc.pdf}   % 700x230
\caption{Examples of composition associations.}
\label{fig:compositionAssoc}
\end{figure}


\index{profile@\texttt{Profile}}
\index{user@\texttt{User}}
Consider the examples in Figure~\ref{fig:compositionAssoc}. In this social
networking site, every \texttt{User} has a \texttt{Profile}. That
\texttt{User} is the \textit{only} one with that particular profile (hence
this is at least aggregation) and what's more, \textit{the \texttt{Profile}
has no meaningful existence without its \texttt{User}.} If the user ever
deletes their account, it wouldn't make sense to have a disembodied
\texttt{Profile} object lying around, so it should automatically disappear as
well. This lifespan connection is really the only difference between the white
diamond and the black.

\index{email@\texttt{Email}}
\index{attachment@\texttt{Attachment}}
On the right-hand side is an example from some kind of email reader
application (like Outlook, gmail, or Thunderbird). A user can compose an
\texttt{Email} with some text and a list of recipients, and then add
\texttt{Attachment}s to it to send images, documents, code, \textit{etc.} But
what if the user decides to abandon the message before sending it? The
\texttt{Email} object should go away, but its \texttt{Attachment}s should too.
Hence this is another example of composition.

\subsubsection{Compositions in code}

Just as with aggregations, there's no simple Java keyword that magically maps
to the idea of ``composition.'' Instead, the presence of the black diamond
suggests to the programmer the intended function of the classes involved, and
she will write the code with this in mind.


\subsection{Association annotations}

\index{annotation}
As if all this weren't enough, there are also a couple more syntactic things
to learn about UML associations. An \textbf{annotation} is another mark on
part of a diagram that gives more detail about how it is to be understood or
implemented. We've already seen two examples of this: the stereotypes we
included next to dependency lines are a type of annotation, as are the
arrowheads to indicate navigability. We'll learn two more in this section.

\subsubsection{Multiplicity}

\index{multiplicity}
The \textbf{multiplicity} of an association indicates \textit{how many}
objects are involved in each concrete relationship. It's important to realize
that even though multiplicity is shown on a class diagram, it's really a
statement about \textit{objects}.

\index{license@\texttt{License}}
\index{association!one-to-one}
\index{one-to-one association}
Let's start with the left-most example in Figure~\ref{fig:multiplicity}. There
we have two classes from a DMV software system, connected with a ``has-a''
association between \texttt{Driver} and \texttt{License}, navigable both ways.
Note the numeral ``\textbf{1}'' annotation both sides of the arrow. This
indicates that \textit{every \texttt{Driver} ``goes with'' just one
\texttt{License} object}, and \textit{every \texttt{License} also goes with
just one \texttt{Driver}.} This is called a \textbf{one-to-one association},
sensibly enough.

\begin{figure}[ht]
\centering
\includegraphics[width=1\textwidth]{multiplicity.pdf}   % 790x270
\caption{Association annotations indicating multiplicity.}
\label{fig:multiplicity}
\end{figure}

\index{star ($\star$)}
\index{association!one-to-many}
\index{one-to-many association}
\index{adventurer@\texttt{Adventurer}}
\index{weapon@\texttt{Weapon}}
In the center example, on the other hand, we have a ``$\star$'' on the side of
the arrow that connects to \texttt{Weapon}. In UML, the symbol ``$\star$''
means \textbf{zero or more}. So here's how we interpret this
\textbf{one-to-many association}: every \texttt{Adventurer} has zero or more
weapons, while every \texttt{Weapon} is possessed by just one
\texttt{Adventurer}. Note that since the direction is only navigable in one
direction, this indicates that although an \texttt{Adventurer} is aware of
which \texttt{Weapon}s she owns, the \texttt{Weapon} objects are \textit{not}
aware of which \texttt{Adventurer} owns them. This knowledge (or lack thereof)
is perfectly okay, and does not invalidate the meaning of the \textbf{1} or
the $\star$ in the slightest.

\index{association!many-to-many}
\index{many-to-many association}
\index{course@\texttt{Course}}
\index{transcript@\texttt{Transcript}}
Finally, on the right side, we have a \textbf{many-to-many association} between
\texttt{Tran\-script} and \texttt{Course}. This says that every
\texttt{Transcript} object is associated with potentially multiple
\texttt{Course} objects, while each \texttt{Course} object appears on more than
one \texttt{Transcript}. In terms of navigability, \texttt{Transcript}s
maintain a record of which \texttt{Course}s they contain, but
\texttt{Course} objects don't know which \texttt{Transcript}s they appear on
(if any).

\index{zero or more (0..$\star$)}
\index{one or more (1..$\star$)}
You'll occasionally see more elaborate multiplicity notations on class
diagrams. The notation ``\textbf{0..$\star$}'' means ``zero or more''...which is
of course exactly what plain old ``$\star$'' means. The only reason for a
designer to write ``\textbf{0..$\star$}'' is for emphasis: she is stressing to
the coding team that an object of the first type may well have \textit{zero}
objects of the second type at any given time; this is a real possibility. In
contrast, if she writes ``\textbf{1..$\star$}'' that means ``one or more,''
which signals the coder ``by the way, every object of the first type should
\textit{always} be assigned to at least one object of the second type; you
should keep that in mind as you code.'' Even more rarely, you'll see
multiplicities like ``\textbf{5}'' (``each object of type A is associated with
exactly five objects of type B''), or ``\textbf{3..8}'' (``each object of type A
is associated with anywhere from three to eight objects of type B''),
\textit{etc.} These are uncommon, especially since as we'll see in the next
section, there really isn't any way to code those constraints explicitly in a
language like Java.



\subsubsection{Multiplicity in code}

\index{instance variable (inst var)}
\index{driver@\texttt{Driver}}
\index{license@\texttt{License}}
\index{adventurer@\texttt{Adventurer}}
\index{weapon@\texttt{Weapon}}
\index{collection}

So what does all this look like in code? Well, first remember that inst vars
are only used in the direction(s) along which the association is navigable. For
Figure~\ref{fig:multiplicity}, this means that only \texttt{Driver},
\texttt{License}, \texttt{Adventurer}, and \texttt{Transcript} will have inst
vars related to these associations; \texttt{Weapon} and \texttt{Course} will
not. Furthermore, if the multiplicity is a \textbf{1}, the inst var will be of
the type the arrow is pointing to; if it's a $\star$, it will be \textit{some
collection} of that type. Which sort of collection is used -- an array, an
\texttt{ArrayList}, a \texttt{Hashtable}\footnote{See
section~\ref{sec:hashtable} (p.~\pageref{sec:hashtable}) if you're unfamiliar
with the \texttt{Hashtable} data type.}, a \texttt{Set},
\textit{etc.}~-- is normally up to the programmer, and is decided based on the
run-time performance features of that collection type.

So here's some code we might reasonably expect to see from our three examples:

\begin{Verbatim}[fontsize=\small,samepage=true,frame=single]
class Driver {               class Adventurer {
    String name;                 String name;
    License license;             int hitPoints;
    ...                          ArrayList<Weapon> weapons;
}                                ...
                             }
class License {
    String number;           class Transcript {
    Driver owner;                Course[] courses;
    ...                          ...
}                            }
\end{Verbatim}

Here the programmer of the \texttt{Adventurer} class has chosen to use an
\texttt{ArrayList} to hold each adventurer's weapons, while the
\texttt{Transcript} author decided on a simple array. In terms of being
faithful to the design, neither choice is right or wrong.

\subsubsection{Roles}

\index{role}
Our last type of association annotation has to do with \textbf{roles}.
Sometimes, a design will be specific not only about the \textit{existence} of
the association between two classes, and about which-knows-about-which, and
about how-many-are-involved, but also the intended \textit{meaning} of the
relationship. In other words, it may specify what role each of the object
types is expected to play with respect to the other. This may sound a bit
abstract, but some examples will make it clearer.

\begin{figure}[ht]
\centering
\includegraphics[width=1\textwidth]{roles.pdf}   % 670x400
\caption{Association annotations indicating roles.}
\label{fig:roles}
\end{figure}

\index{hero@\texttt{Hero}}
\index{villain@\texttt{Villain}}
The upper-left example in Figure~\ref{fig:roles} shows a piece of a Marvel
comic book database application. We have \texttt{Hero} and \texttt{Villain}
classes, and a one-to-one association between them...but what does the
association \textit{mean}? If \texttt{Hero X} ``goes with'' \texttt{Villain Y},
does that mean that \texttt{X} has recently beaten up \texttt{Y}? That
\texttt{X} admires \texttt{Y}? That \texttt{X} secretly \textit{is}
\texttt{Y}, unbeknownst to the public?

\index{archnemesis}
The word ``archnemesis'' next to the \texttt{Villain}-side of the arrow spells
it out. It's called a \textbf{role name}. It tells us that in this
relationship, the role that the \texttt{Villain} plays with respect to the
\texttt{Hero} is that the former is the archenemy of the latter.

\index{twitteruser@\texttt{TwitterUser}}
\index{X@``X'' (Twitter)}

Moving to the right side of the diagram, we have an interesting situation
involving only one class: \texttt{TwitterUser}\footnote{Twitter is the former
name of the social networking site that Elon Musk renamed ``X.''}. This class
apparently has an association to itself! This turns out not to be as weird as
it might seem. In fact, if you think about a social network like Twitter (or
``X''), the most meaningful relationships \textit{are} between objects of the
same class. And that's the key to de-weirding it in your mind: remember that an
association is a statement about \textit{objects}, not classes. We're not
saying ``\texttt{TwitterUser} has a relationship with itself'' but rather
``each \texttt{TwitterUser} is related to zero or more other
\texttt{TwitterUser}s.''

And what do those relations mean, you ask? The role name tells us: one of the
users ``follows'' the other in the X sense. In this diagram, we have role
names on both sides of the arrow, although that's probably not strictly
necessary. What is interesting here is the navigability of the association:
according to the design, a \texttt{TwitterUser} object is aware of what other
\texttt{TwitterUser}s follow him/her, but not which \texttt{TwitterUser}s
he/she follows. If the design team decided they needed to track that
separately, they'd need another arrowhead on the top side of the line.

\index{student@\texttt{Student}}
\index{professor@\texttt{Professor}}
Finally, the bottom example illustrates two \textit{different} associations
between the same two classes. This can happen as well. In this case, there are
two distinct roles that \texttt{Professor}s play with respect to
\texttt{Student}s: as their instructors (each student has several) and as
their advisor (each student has one). The role names are imperative here,
since otherwise the programming team would be lost as to why there are two
relationships and what each one is supposed to mean.

\subsubsection{Roles in code}

Often, the role name on the diagram is simply used as the instance variable
name in the code. For instance, I'd expect to see something like this:

\vspace{-.15in}
\begin{Verbatim}[fontsize=\small,samepage=true,frame=single]
class Student {
    String major;    
    Professor advisor;
    ArrayList<Professor> instructors;
}
\end{Verbatim}
\vspace{-.15in}

since those names were handed to us on a silver platter in the design diagram.

\section{Visibility}

\index{visibility}
The other parts of UML class diagrams that we'll annotate with extra
information will indicate the level of \textbf{visibility} that the designer
intends the various inst vars, methods, and even classes to possess.
Visibility has to do with promoting \textbf{encapsulation}, the most important
of all OO principles as we learned in Chapter~\ref{ch:encapsulation}.

This is one area where our UML diagrams, ostensibly
programming-language-neutral, will betray a very Java-ish flavor. That's
because in Java, there are four specific visibility levels for methods and
inst vars (two for classes), each with a precise meaning, and we'll have UML
syntax to indicate each. The complete list can be found in the tables in
Figures~\ref{fig:visibilityLevels} and \ref{fig:visibilityLevelsClasses}. In
both tables, the visibility levels are listed in order from most restrictive
to least restrictive.

\subsection{Java packages}
\label{sec:packages}
\index{package}

Now's as good a time as any to mention the notion of Java ``packages.'' This
was a language innovation intended to provide an organizational mechanism:
related classes can be grouped together into a construct called a
\textbf{package}. This really isn't much more than being able to store
\texttt{.java} files in different directories to keep them organized, except
for one thing: the language itself is aware of which Java classes are members
of which packages, and can enforce visibility based on that notion, as we'll
see below. For now, here's the basics about packages:

\begin{compactenum}

\item Package names can be a single word (like ``\texttt{combat}'') or a
dot-separated sequence of words (like
``\path{com.gearbox.halo.simulator.combat}'').

\item The dot-separated-sequence variety is kinda sorta meant to convey a
hierarchy, from general to specific. In the previous example -- from the
\textit{Halo} videogame created by Gearbox Software -- ``\texttt{combat}'' is a
subset of ``\texttt{simulator},'' which is part of the ``\texttt{halo}''
program designed at ``\texttt{gearbox.com}''. (Since ``\texttt{com}'' is more
general than ``\texttt{gearbox}'' -- just like ``\texttt{edu}'' is more general
than ``\texttt{umw}'' -- many package names begin with a domain name written in
reverse order like this.)

\item However, even though it looks like a hierarchy, \textit{Java has no
notion of subpackages.} In other words, although the
``\path{com.gearbox.halo.simulator}'' package looks like it would be a
``subpackage'' of ``\path{com.gearbox.halo},'' in actual fact it is not. It's
just a naming convention, and there's no way (for example) to ``import
everything from \path{com.gearbox.halo} on down.''

\item Every class is in one (and only one) package. This must be specified in
\textit{both} of two ways: (1) the first (non-comment) line of the file must be
a \textbf{package declaration} like ``\texttt{package}
\path{com.gearbox.halo.simulator.combat}'', and (2) the \texttt{.java} file
itself must be physically in a directory called
``\path{com/gearbox/halo/simulator/combat}.''\footnote{If you forget either one
of these two things, or make them incompatible with each other, your code will
be officially unreachable by any Java program. (Yes, that was a dumb design
decision on Java's part.)}

\item If a class has no \texttt{package} statement, then it is considered to be
in ``the default package,'' which just means ``the package with no name.''
\end{compactenum}

\subsection{Visibility levels}

Okay, back to visibility. Let's look at the syntax and the operational
implications of the different visibility levels in
Figures~\ref{fig:visibilityLevels} (for methods and inst vars) and
\ref{fig:visibilityLevelsClasses} (for classes themselves).

\index{private@\texttt{private}}
\index{protected@\texttt{protected}}
\index{public@\texttt{public}}
\index{package visibility}
\begin{figure}[ht]
\centering
\small
\begin{tabular}{c|c|c|l}
\thead{visibility level} & \thead{Java keyword} & \thead{UML} &
\thead{visible to...} \\
\hline
private & \texttt{private} & \textbf{--} & the class itself \\
package & (none) & \textbf{\freakingtilde} & any class from same package\\
protected & \texttt{protected} & \textbf{\#} & the same
package, or subclass\footnote{A ``subclass'' has to
do with the topic of \textbf{inheritance} in OO, which we will cover in gory
detail in Chapters~\ref{ch:inheritance} and \ref{ch:inheritance2}.
For now, I just want to make the table complete.} \\
public & \texttt{public} & \textbf{$\plus$} & any method anywhere \\
\end{tabular}
\vspace{.1in}
\caption{The four Java visibility levels for \textit{methods and inst vars}.}
\label{fig:visibilityLevels}
\normalsize
\end{figure}

\index{plus (+)}
\index{minus (-)}
\begin{figure}[h]
\centering
\small
\begin{tabular}{c|c|c|l}
\thead{visibility level} & \thead{Java keyword} & \thead{UML} &
\thead{visible to...} \\
\hline
non-public & (none) & (none) & only classes in same package \\
public & \texttt{public} & \textbf{$\plus$} & any class anywhere \\
\end{tabular}
\vspace{.2in}
\caption{The two Java visibility levels for \textit{classes}.}
\vspace{.3in}
\label{fig:visibilityLevelsClasses}
\normalsize
\end{figure}

\pagebreak
Now it's important to understand that unlike multiplicity, visibility
modifiers make a statement about \textit{classes}, not \textit{objects}. Also,
crucially, visibility is about the very \textit{existence} of the method, inst
var, or class, not its \textit{value.} This is very commonly misconstrued, so
let me clarify with an example.

\medskip
\index{ballplayer@\texttt{Ballplayer}}
Suppose a class diagram included the class in
Figure~\ref{fig:visAboutClasses}. Here, for the first time, we see visibility
modifiers in action. In particular, the \texttt{numHits} and
\texttt{numAtBats} inst vars are both marked as private, while the
\texttt{isBetterThan()} method is public.

\begin{figure}[h]
\centering
\bigskip
\includegraphics[width=0.8\textwidth]{visAboutClasses.pdf}
\medskip
\caption{A class whose components bear visibility annotations.}
\label{fig:visAboutClasses}
\bigskip
\end{figure}

\pagebreak
\index{client code}
Here's the kind of client code we want to make possible with this method:

\vspace{-.15in}
\begin{Verbatim}[fontsize=\footnotesize,samepage=true,frame=single]
    ...
    Ballplayer jeter = new Ballplayer("Jeter");
    Ballplayer arod = new Ballplayer("Rodriguez");
    if (jeter.isBetterThan(arod)) {
        System.out.println("Sign Jeter to a zillion dollars!");
    }
    ...
\end{Verbatim}

Let's inspect the \textit{inside} of the \texttt{.isBetterThan()} method
(\textit{i.e.}, its implementation). Suppose it reads like
this\footnote{Apologies to baseball fans for the gross simplification of
reducing an entire player's ``goodness'' down to his or her batting average. Of
course in real life there are all kinds of other stats that come into play
here -- slugging percentage, base running stats, defensive ability,
\textit{etc.} -- as well as impossible-to-quantify aspects like teamwork,
inspiration, and clubhouse chemistry.}:

\begin{Verbatim}[fontsize=\footnotesize,samepage=true,frame=single]
class Ballplayer {
    private int numHits;
    private int numAtBats;
    ...
    public boolean isBetterThan(Ballplayer other) {
        double myBA = ((double)this.numHits)/this.numAtBats;
        double otherBA = ((double)other.numHits)/other.numAtBats;
        if (myBA > otherBA) {
            return true;
        } else {
            return false;
        }
    }
    ...
}
\end{Verbatim}

Now the key line I want to draw your attention to is the \textit{second} line
of the method. It reads:

\begin{verbatim}
double otherBA = ((double)other.numHits)/other.numAtBats;
\end{verbatim}

My question to you, dear reader, is this: do you think this line ought to
compile without errors, or no? Take a moment to consider your answer.

Many, many students assume this line will \textit{not} compile cleanly. Here's
their reasoning: ``We're calling \texttt{.isBetterThan()} on a particular
\texttt{Ballplayer} object (say, \texttt{jeter}). And we're passing another
\texttt{Ballplayer} object as a parameter (say, \texttt{arod}). Now both
\texttt{numHits} and \texttt{numAtBats} are marked \textbf{\texttt{private}}
in the class. Therefore, \texttt{arod}'s values for these should be protected
from, and unavailable to, the \texttt{jeter} object. It stands to reason that
this will not be allowed. Otherwise, we'd be allowing one object to access
another object's private data.''

This sounds so eminently reasonable, and yet it is dead wrong. Here's why:

\index{private@\texttt{private}}
\begin{itemize}
\itemsep.1em
\item[{\color{darkred} \XSolidBold}] A ``private'' inst var does \underline{not} mean that one object's
\textit{value} is hidden from another \textit{object}.
\item[{\color{darkgreen} \CheckmarkBold}] A ``private'' inst var \underline{does} mean that the very
\textit{existence} of one class's inst var is hidden from other
\textit{classes}.
\end{itemize}

\index{encapsulation}
In other words, it's not a ``data privacy'' thing like keeping your information
inaccessible to creepy people online. Instead, it's a \textit{code
encapsulation thing} that prevents one class from making (and thereafter
depending upon) assumptions about another class's design decisions. In terms
of the online creep example, here's how I'd explain it:

\begin{itemize}
\itemsep.1em
\item[{\color{darkred} \XSolidBold}] Making \texttt{SSN} (Social Security Number) a private inst var of the
\texttt{Person} class does \underline{not} mean that one \texttt{Person} object
cannot find out another \texttt{Person} object's SSN.
\item[{\color{darkgreen} \CheckmarkBold}] Making \texttt{SSN} a private inst var of the \texttt{Person} class
\underline{does} mean that \texttt{Dog}s, \texttt{Website}s,
\texttt{CreditCard}s, \textit{etc.}~don't even know that people \textit{have}
Social Security Numbers.
\end{itemize}

In yet other words, visibility is about \textit{variables} and
\textit{classes}, not \textit{values} and \textit{objects}.

The code above does compile cleanly for one simple reason: it's a method of the
\texttt{Ballplayer} class. Any method of the \texttt{Ballplayer} class can talk
about any inst var or method of the \texttt{Ballplayer} class, regardless of
which particular object is in view.

To complete the example, here's some code which indeed does \textit{not}
compile because of those private \texttt{numHits} and \texttt{numAtBats}:

\begin{Verbatim}[fontsize=\footnotesize,samepage=true,frame=single]
class Team {
    private ArrayList<Ballplayer> roster;
    ...    
    public void printRoster() {
        System.out.println("Name          Hits    ABs");
        for (Ballplayer b : roster) {
            System.out.println(b.name + "     " + b.numHits +
                "  " + b.numAtBats);
        }
    }
    ...
}
\end{Verbatim}

\index{team@\texttt{Team}}
\begin{samepage}
When we try to compile it, the \texttt{println()} statement inside the
\texttt{for} loop barfs with:
\footnotesize
\begin{verbatim}
Team.java:25: error: numHits has private access in Ballplayer
    System.out.println(b.name + "     " + b.numHits +
                                           ^
Team.java:26: error: numABs has private access in Ballplayer
        " " + b.numABs);
               ^
\end{verbatim}
\normalsize
\vspace{-.2in}
as we would expect. It's because the offending code is a \texttt{Team} method,
not a \texttt{Ballplayer} method, and therefore cannot refer to any of
\texttt{Ballplayer}'s private components (inst vars or methods).
\end{samepage}

The same mechanic is at play with methods as it is with inst vars: no code in
a class can \textit{call} a method unless it has visibility to that method,
as specified in the rightmost column of Figure~\ref{fig:visibilityLevels}.

If you're wondering why it would ever make sense to have a private method, the
answer is: as a helper method, for other (perhaps public) methods of that
class to call internally. Having lots of short methods to perform basic tasks,
but not exposing those methods outside the class, is one sign of a good
designer.

\subsection{Which visibility level to choose}

\index{visibility}
Both inst vars and methods can technically have any of the four visibility
levels assigned to them from the table in Figure~\ref{fig:visibilityLevels}.
Here are the rules (and strong suggestions) to keep in mind:

\begin{enumerate}
\itemsep.1em
\item Always make all instance variables \texttt{private}\footnote{Or
possibly \texttt{protected}, if you intend to inherit from the class. See
Chapter~\ref{ch:inheritance2}.}. That's the easiest
design decision you'll ever make. \textbf{Public instance variables
unacceptably sacrifice encapsulation.}
\item Always make methods ``as private as possible.'' This promotes
encapsulation and reduces dependencies. When in doubt, err on the side of the
\textit{higher} entry in Figure~\ref{fig:visibilityLevels}, not the lower. If
it turns out you must make it more accessible later on, you can always move
its visibility lower on the chart without breaking anything. The reverse is
not true.
\end{enumerate}

\index{package visibility}
\index{Gosling, James}
Lastly, a word about package-level visibility. There may be design decisions
you make (namely, certain methods you create on a class) that you don't
necessarily want to make publicly accessible to all users of the class, yet
which it does make sense to make available to the other classes that are
collaborating with that class. Package-level visibility was designed for this
purpose. Note that there is no Java keyword for it: it's the default. This is
because Gosling \& Co.~(the designers of Java) were proud of the package
concept and wanted to promote its use among Java developers as much as
possible. So you have to explicitly type if you want any other choice. I think
package-level visibility is a neat feature, but is underutilized.

\subsection{Class visibility}

As shown in Figure~\ref{fig:visibilityLevelsClasses}, the notion of visibility
also extends to entire classes in Java. But it's simpler: either a class is
public, or it's not. If it's public, any class anywhere can refer to it, and
if it's ``non-public'' (yep, that's actually the term) it effectively has
package-level visibility (\textit{i.e.}, only other classes in its package can
use it.)

\index{non-public classes}
Non-public classes thus play the same sort of role as private helper methods
do: the public classes use them to help get their job done, but the non-public
ones aren't designed to be directly instantiated (or even seen) by the outside
world. Their use in practice is somewhat rarer than private methods, but I
encourage their use.


\section{Putting it all together}

All right, let's close this chapter with a small but still full-blown class
diagram that illustrates most of the above features. See if you can interpret
all of Figure~\ref{fig:fullClassDiag} correctly.

\begin{figure}[ht]
\centering
\includegraphics[width=1.05\textwidth]{fullClassDiag.pdf}   % 788x440
\smallskip
\caption{A full-blown class diagram. (The color is not part of UML; I only
colored certain elements so I could refer to them in the text.)}
\label{fig:fullClassDiag}
\end{figure}

\pagebreak
Here's an incomplete list of things we know from the diagram. Each item's
color corresponds to an item in Figure~\ref{fig:fullClassDiag}:

\begin{enumerate}
\itemsep.1em

\definecolor{darkgreen}{rgb}{0,.65,0}
\definecolor{darkblue}{rgb}{0,0,.95}
\item \textcolor{darkgreen}{The \texttt{Ballplayer} class is public, and thus
can be used by any class in any other package. The \texttt{Simulator} class
isn't, though, and can only be referenced by classes in the same package.}

\item \textcolor{BurntOrange}{Although anyone can use a \texttt{Team} object,
only classes in this package can instantiate one. (And to do so, you must
specify a city and a mascot.)} 

\item \textcolor{Turquoise}{Any method that gets its hands on a
\texttt{Ballplayer} can find out his/her age. But only methods of the
\texttt{Ballplayer} class itself can change his/her age.}

\item \textcolor{Red}{Every \texttt{Team} object will have a private instance
variable\footnote{Notice that the ``\textbf{--}'' immediately before the word
``roster'' is a visibility modifier, indicating that the inst var that results
from this association will be private.} called \texttt{roster} which holds a collection (perhaps an
\texttt{ArrayList}) of \texttt{Ballplayer} objects. Each of those
\texttt{Ballplayer}s belongs to only a single \texttt{Team} object, but is not
aware of which \texttt{Team} object that is (\textit{i.e.},
\texttt{Ballplayer} objects don't have an inst var of type \texttt{Team})}.

\item \textcolor{darkblue}{There is a single integer variable \texttt{numTeams}
which is shared among all objects of type \texttt{Team}. It is not visible to
any other class.}

\item \textcolor{Purple}{Somewhere in the static \texttt{main()} method of the
\texttt{Simulator} class we would expect to find code like this:
``\texttt{new Ballplayer(} \texttt{someName, someAge)}''.}

\item \textcolor{Brown}{A \texttt{Simulator} holds on to some number of
\texttt{Team} objects, probably in an instance variable, and each of those
\texttt{Team}s belong only to it.}

\end{enumerate}

Did you pick all those things out? If so, you can read a blueprint, and I
foresee many beautiful buildings in your future!
